% Options for packages loaded elsewhere
\PassOptionsToPackage{unicode}{hyperref}
\PassOptionsToPackage{hyphens}{url}
%
\documentclass[
  12pt,
]{article}
\usepackage{lmodern}
\usepackage{amssymb,amsmath}
\usepackage{graphicx}
\usepackage{ifxetex,ifluatex}
\ifnum 0\ifxetex 1\fi\ifluatex 1\fi=0 % if pdftex
  \usepackage[T1]{fontenc}
  \usepackage[utf8]{inputenc}
  \usepackage{textcomp} % provide euro and other symbols
\else % if luatex or xetex
  \usepackage{unicode-math}
  \defaultfontfeatures{Scale=MatchLowercase}
  \defaultfontfeatures[\rmfamily]{Ligatures=TeX,Scale=1}
  \setmainfont[]{Hoefler Text}
  \setsansfont[]{Gill Sans}
\fi
% Use upquote if available, for straight quotes in verbatim environments
\IfFileExists{upquote.sty}{\usepackage{upquote}}{}
\IfFileExists{microtype.sty}{% use microtype if available
  \usepackage[]{microtype}
  \UseMicrotypeSet[protrusion]{basicmath} % disable protrusion for tt fonts
}{}
\makeatletter
\@ifundefined{KOMAClassName}{% if non-KOMA class
  \IfFileExists{parskip.sty}{%
    \usepackage{parskip}
  }{% else
    \setlength{\parindent}{0pt}
    \setlength{\parskip}{6pt plus 2pt minus 1pt}}
}{% if KOMA class
  \KOMAoptions{parskip=half}}
\makeatother
\usepackage{xcolor}
\IfFileExists{xurl.sty}{\usepackage{xurl}}{} % add URL line breaks if available
\IfFileExists{bookmark.sty}{\usepackage{bookmark}}{\usepackage{hyperref}}
\hypersetup{
  pdftitle={English Department Undergraduate Courses},
  hidelinks,
  pdfcreator={LaTeX via pandoc}}
\urlstyle{same} % disable monospaced font for URLs
\usepackage[paperheight=8.5in,paperwidth=5.5in,bottom=0.75in,top=0.75in,left=0.5in,right=0.5in]{geometry}
\setlength{\emergencystretch}{3em} % prevent overfull lines
\providecommand{\tightlist}{%
  \setlength{\itemsep}{0pt}\setlength{\parskip}{0pt}}
\setcounter{secnumdepth}{-\maxdimen} % remove section numbering
\ifluatex
  \usepackage{selnolig}  % disable illegal ligatures
\fi


\newlength{\cslhangindent}
\setlength{\cslhangindent}{1.5em}
\newlength{\csllabelwidth}
\setlength{\csllabelwidth}{3em}
\newlength{\cslentryspacingunit} % times entry-spacing
\setlength{\cslentryspacingunit}{\parskip}
\newenvironment{CSLReferences}[2] % #1 hanging-ident, #2 entry spacing
 {% don't indent paragraphs
  \setlength{\parindent}{0pt}
  % turn on hanging indent if param 1 is 1
  \ifodd #1
  \let\oldpar\par
  \def\par{\hangindent=\cslhangindent\oldpar}
  \fi
  % set entry spacing
  \setlength{\parskip}{#2\cslentryspacingunit}
 }%
 {}
\usepackage{calc}
\newcommand{\CSLBlock}[1]{#1\hfill\break}
\newcommand{\CSLLeftMargin}[1]{\parbox[t]{\csllabelwidth}{#1}}
\newcommand{\CSLRightInline}[1]{\parbox[t]{\linewidth - \csllabelwidth}{#1}\break}
\newcommand{\CSLIndent}[1]{\hspace{\cslhangindent}#1}

\usepackage{multicol}


\title{Somewhere in New Jersey\ldots{}}
\author{}
\date{}

\begin{document}

In a \href{/2013/08/teaching-dh}{previous post} on a graduate course I
taught last Spring, I mentioned the server I ended up using a way to try
to establish some uniformity of access to software packages and tools.
In this post, I'll try to add a few details.

\hypertarget{virtual-private-servers}{%
\section{Virtual Private Servers}\label{virtual-private-servers}}

The first thing to understand about ``the server'' is that I rented a
\emph{VPS}, a ``virtual(ized) private server,'' rather than some shared
space. If you don't know what that means, let me try to explain (if you
do know what that means, you may want to move along to the next
section).

The word \emph{server} itself is one of the slipperier bits of our
contemporary argot. It can name a number of different links in a
relationship between one piece of software and another (to say nothing
of the people on either end, or perhaps in between). This slipperiness
is only compounded with the advent of \emph{virtual private servers}.

Prior to virtualization, if you wanted to host a webpage, you could
either run (or rent) your own dedicated hardware, or you go with a
``shared hosting'' option. For most folks who run their own blog, shared
hosting remains the obvious choice. This blog is hosted on such a shared
host. Shared hosting relies on a few facts:{\textbf{LAMP}: that is Linux
(an operating system to manage the hardware resources, to schedule
processes\ldots{} you know, to turn electricity, plastic, and rare earth
elements into a computer), Apache (the most widely used web server),
MySQL (the most widely used database; which has its own database
\emph{server}), and (usually) PHP (a scripting language).}

\begin{itemize}
\tightlist
\item
  most folks' hosting needs can be solved by a single ``stack'' of
  common software, usually the so-called \textbf{LAMP} stack
\item
  most folks' blogs don't get sufficient traffic to require that much
  hardware, and so many blogs (websites, whatever) can be ``served'' by
  a single (``shared'') host.
\item
  the OS can easily separate out different users; that way, I can run
  Wordpress and create a database, and you can run Drupal and create a
  database, all on the same hardware, without either of us being able to
  destroy one another's data.
\end{itemize}

The drawback to shared hosting is that you have very little control over
the server. You usually can't, for instance, install new software,
beyond packages of PHP scripts. Of course, for web hosting, that's no
big deal; you don't really need to install anything beyond whatever CMS
you want anyway. {Anyone who has had problems with a web host's version
of PHP, or similar issues, knows that this isn't quite true.}

This compromise with respect to control over the server configuration is
a function of cost (and of course, expertise; do you really want to have
to worry about building that ``software stack''?). To have complete
control over a server would mean that you had a dedicated server (in
this case, ``server'' means actual \emph{hardware}), perhaps in your
office (in theory it could be your laptop), in an IT closet, or rented
in a server farm somewhere. With the advent of software virtualization,
however, this changed. With virtualization, a single piece of hardware
can run multiple ``virtual machines''; the host system simulates another
machine, on which you can then run software which itself doesn't really
know the difference. (If you run Windows on a Mac with Parallels, this
idea may be quite familiar.) You can do the same thing with a server.
That computer that talked to the internet, and served web pages (or
whatever), is not just a \emph{virtual} machine. And such a virtualized
private server (a VPS, as opposed to a \emph{shared} host) reconfigures
the costs involve, and makes it more affordable to give you more control
over the software installed, without incurring the full cost of
owning/renting/running an actual physical hardware server.

A virtualized server is (\emph{very} significantly) cheaper than running
your own hardware, but allows you many of the advantages of actually
running your own hardware. It is still, in general, more expensive than
shared hosting. To give you some sense of the cost, the lowest tier
\href{https://www.linode.com/}{Linode} VPS is \$20/month.
\href{https://www.digitalocean.com/pricing}{Digital Ocean's lowest tier}
is only \$5/month.

\hypertarget{a-customized-server}{%
\section{A Customized Server}\label{a-customized-server}}

Because it is so customizable, I thought a VPS could offer a solution to
the challenge of trying to make software uniformly available to students
enrolled in the digital humanities seminar I was teaching. With such a
system I could give everyone access to Python and the NLTK (and its
associated corpora) without having to ask folks to install that software
on their own machines. I could install MALLET and R (and relevant R
packages) and Stanford's Named Entity Recognizer and an XSLT processor.
This was also a relatively flexible solution; if someone wanted to try
something else, perhaps something I'd never heard of, it was in many
easy to install it on the server.

The \emph{easy} in that last sentence is a function of Linux package
management. If you're used to installing from .EXEs (or .DMGs) you
download from the web, the world of package management can seem arcane.
However, for large pieces of software, package management systems are
wonderful and can be (deceptively) simple to use. While getting
everything up and running is a bit of a trick (you need to first install
an operating system---about which a little more below---and then some
basic software to let you connect to the server), installing a piece of
software, like the R language, is as simple as typing:

\{\%highlight bash \%\} sudo apt-get install r-base \{\%endhighlight
\%\}

And then, you have R installed, and you're ready to go at the command
line. In my experience, Linux package management is often \emph{easier}
than trying to handle software dependencies on other OSes (at least once
you're familiar with conventions of your package management system).

\emph{Easy} is also relative; if you're comfortable at the command line,
using a package manager feels intellectually more intuitive and
comprehensible than dragging a DMG into an ``Applications'' folder, or
double-clicking an EXE. But if you're uncomfortable with the command
line, this will likely feel as uncomfortable as navigating your
filesystem or anything else.

\hypertarget{server-setup}{%
\section{Server Setup}\label{server-setup}}

Before you can install packages, though you need to first install a
base, Linux operating system. If you're unfamiliar with Linux, this may
not be the kindest or easiest way to get acquainted with it, though the
good folks at Linode (hardly unbiased) insist
\href{https://www.linode.com/faq.cfm}{``If you're looking to learn,
there is no better environment. Experiment with the different Linux
flavors, redeploy from scratch in a matter of minutes.''}. I have spent
too much of my life playing with Linux distros, and so this part of the
process felt quite natural. And yet, I \emph{still} managed to make what
I now consider a wrong choice in configuring the server. Linux comes in
a wide array of flavors or distributions.{Technically, ``Linux'' is not
an OS, but an OS ``kernel.'' This distinction, and what we call things,
can get
\href{http://en.wikipedia.org/wiki/GNU/Linux_naming_controversy}{contentious}.}
Of the options Linode offers (Ubuntu, Arch, OpenSuse, Gentoo, CentOS,
Slackware, Debian, Fedora) I opted for Debian. Debian has a reputation
for being a very stable distro; and so it would be a great choice for
running a web-server. Of course, I \emph{wasn't} running a web and so
stability was not, in fact, my \emph{chief} concern. I probably should
have chosen a distribution which prioritized not stability, but the ease
and availability of new and up-to-date software packages. Arch Linux,
with its ``rolling release'' schedule (and the operating system
\href{http://www.mylinuxrig.com/post/11831613369/the-linux-setup-chris-forster-academic}{I
once loved} with a passion I've not since been able to match) would have
made more sense. It would have been \emph{easier} with Arch to install
the most update versions of certain Python packages, etc etc. Oh well.
Maybe next time.

Once you're base OS is installed; it's time to install the basic
packages you will need to do \emph{anything at all}. But since you now
are responsible for a computer somewhere in New Jersey{I must say, the
responsiveness of the Linode servers shocked me; that computer in NJ was
consistently more responsive than my home media server. I regularly ran
emacs sessions \emph{on the server} and found them completely
responsive.} you need to worry about security. You don't want someone
hijacking your VPS and using to send spam or whatever else. Linode
offers \href{http://library.linode.com/securing-your-server}{some tips}
and I consulted \href{http://feross.org/how-to-setup-your-linode/}{this
page} for some advice as well. It wasn't nearly as scary as I imagined.
I installed \texttt{fail2ban} (and left the default settings), turned
off root log-in from ssh, and that was about it. I have (so far) had no
problems.

\hypertarget{connecting-and-interacting-with-the-server}{%
\section{Connecting and Interacting with the
Server}\label{connecting-and-interacting-with-the-server}}

I've sort of glossed over a pretty fundamental fact; with the exception
of RStudio server (mentioned below), the only way to interact with the
server as I've described it is through
\href{http://en.wikipedia.org/wiki/Secure_Shell}{SSH}; and so the only
access you have to the server is command line access. For the most part,
that was fine for the goals I had for the class. Command line access
allowed people to experiment with Python and NLTK; they could run
MALLET, and similar things. You couldn't run, say, Gephi on the server
though.

And this limitation proved frustrating for folks working with film and
images. For one thing, moving large movie/image files back and forth to
the server would have proved unpleasant; moreover, software like
\href{http://rsbweb.nih.gov/ij/}{ImageJ} couldn't be installed on the
server and had to be installed locally. (I did manage, though, to do
some image manipulation with ImageMagick---here is every page of the
\emph{Little Review} in a single image (a larger, \textasciitilde150M,
image is available
\href{https://www.dropbox.com/s/idnsh1tmkteeevu/lr.png}{here}):

For folks who were interested in using \texttt{R},
\href{http://www.rstudio.com/ide/docs/server/getting_started}{RStudio
Server} worked \emph{wonderfully}; it allowed folks to connect to the
server through their web browser and have an RStudio Session that looked
something like this:

While I gripe about \texttt{R}, RStudio is really excellent. If there is
comparable server project that will let folks run python and python
packages (including matplotlib) through a browser in a similar way,
please, \emph{please} let me know. RStudio provides a great way to
provide a standardized R environment with a common set of shared
packages (and perhaps even data) available to all.

There are other things one could do with sort of set-up; if you wanted
to run an old-school Bulletin Board System or MUD, you could do it (here
are \href{http://lunduke.com/?p=2156}{some ideas} about running a BBS
system). {You also might look at
\href{http://www.telnetbbsguide.com/ssh.htm}{The BBS Corner's Telnet \&
Dial-Up BBS Guide} or \href{http://www.convolution.us/}{Convolution
BBS}.} As a way to offer \emph{certain} pieces of software to people
without requiring them to install them, running this server was very
helpful. If you're doing tasks that can be scripted and which can often
take a very long time to complete (such as topic modeling, or named
entity extraction, or POS tagging){I would add certain types of image
manipulation---as I did with the \emph{Little Review} example noted
above; though as I say, shuttling the images back and forth can be
unpleasant; this unpleasantness is partially allayed if you're scripting
your image acquisition with a script, ``a little \texttt{wget} magic'',
or similar ought to work.} you can set them going, disconnect from the
server and then check on them later (using a program like
\href{http://www.gnu.org/software/screen/}{\texttt{screen}} to make this
easy; it's \emph{really} great to be walking to home, knowing that
somewhere in New Jersey a computer is dutifully seeking 100 topics
across 5000 documents).

I've skipped over some of the other unglamorous, sysadminy things one
must do: creating user accounts for each person in the class; creating a
shared directory that everyone could read and write to; and other things
like that. For someone comfortable at the command line, and interested
to learn, all of that stuff is entirely manageable.

\end{document}
