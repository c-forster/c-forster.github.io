% Options for packages loaded elsewhere
\PassOptionsToPackage{unicode}{hyperref}
\PassOptionsToPackage{hyphens}{url}
%
\documentclass[
  12pt,
]{article}
\usepackage{lmodern}
\usepackage{amssymb,amsmath}
\usepackage{graphicx}
\usepackage{ifxetex,ifluatex}
\ifnum 0\ifxetex 1\fi\ifluatex 1\fi=0 % if pdftex
  \usepackage[T1]{fontenc}
  \usepackage[utf8]{inputenc}
  \usepackage{textcomp} % provide euro and other symbols
\else % if luatex or xetex
  \usepackage{unicode-math}
  \defaultfontfeatures{Scale=MatchLowercase}
  \defaultfontfeatures[\rmfamily]{Ligatures=TeX,Scale=1}
  \setmainfont[]{Hoefler Text}
  \setsansfont[]{Gill Sans}
\fi
% Use upquote if available, for straight quotes in verbatim environments
\IfFileExists{upquote.sty}{\usepackage{upquote}}{}
\IfFileExists{microtype.sty}{% use microtype if available
  \usepackage[]{microtype}
  \UseMicrotypeSet[protrusion]{basicmath} % disable protrusion for tt fonts
}{}
\makeatletter
\@ifundefined{KOMAClassName}{% if non-KOMA class
  \IfFileExists{parskip.sty}{%
    \usepackage{parskip}
  }{% else
    \setlength{\parindent}{0pt}
    \setlength{\parskip}{6pt plus 2pt minus 1pt}}
}{% if KOMA class
  \KOMAoptions{parskip=half}}
\makeatother
\usepackage{xcolor}
\IfFileExists{xurl.sty}{\usepackage{xurl}}{} % add URL line breaks if available
\IfFileExists{bookmark.sty}{\usepackage{bookmark}}{\usepackage{hyperref}}
\hypersetup{
  pdftitle={English Department Undergraduate Courses},
  hidelinks,
  pdfcreator={LaTeX via pandoc}}
\urlstyle{same} % disable monospaced font for URLs
\usepackage[paperheight=8.5in,paperwidth=5.5in,bottom=0.75in,top=0.75in,left=0.5in,right=0.5in]{geometry}
\setlength{\emergencystretch}{3em} % prevent overfull lines
\providecommand{\tightlist}{%
  \setlength{\itemsep}{0pt}\setlength{\parskip}{0pt}}
\setcounter{secnumdepth}{-\maxdimen} % remove section numbering
\ifluatex
  \usepackage{selnolig}  % disable illegal ligatures
\fi


\newlength{\cslhangindent}
\setlength{\cslhangindent}{1.5em}
\newlength{\csllabelwidth}
\setlength{\csllabelwidth}{3em}
\newlength{\cslentryspacingunit} % times entry-spacing
\setlength{\cslentryspacingunit}{\parskip}
\newenvironment{CSLReferences}[2] % #1 hanging-ident, #2 entry spacing
 {% don't indent paragraphs
  \setlength{\parindent}{0pt}
  % turn on hanging indent if param 1 is 1
  \ifodd #1
  \let\oldpar\par
  \def\par{\hangindent=\cslhangindent\oldpar}
  \fi
  % set entry spacing
  \setlength{\parskip}{#2\cslentryspacingunit}
 }%
 {}
\usepackage{calc}
\newcommand{\CSLBlock}[1]{#1\hfill\break}
\newcommand{\CSLLeftMargin}[1]{\parbox[t]{\csllabelwidth}{#1}}
\newcommand{\CSLRightInline}[1]{\parbox[t]{\linewidth - \csllabelwidth}{#1}\break}
\newcommand{\CSLIndent}[1]{\hspace{\cslhangindent}#1}

\usepackage{multicol}


\title{"you can't eat the food porn": The Obscene, the Pornographic, the
Excessive, the Purposeless}
\author{}
\date{}

\begin{document}

as long as sexuality remains as integrated into social life in general
as say, eating, its possibilities of symbolic extension are to that
degree limited ---Jameson

The notorious difficulty of defining \emph{obscenity} and
\emph{pornography} has a long history. But the term \emph{porn} has,
over the past decade or so, begun to crop up with new meanings, in new
places. This post assumes that this accretion is not merely random and
is subject to at least some degree of analysis and consideration and, in
that spirit, tries to sort that meaning out. Be forewarned, however,
this post will feature nothing that is itself prurient{Prurience is of
key importance as a criterion in the legal definition of obscenity.
``Pornographic'' and ``pornography,'' unlike \emph{obscene} and
\emph{obscenity}, do not have strict legal definitions.}; indeed, the
meanings of \emph{porn} are now freed from sexual explicitness, at the
exact historical moment which has seen a relatively broad acceptance of
a distinct and well-defined pornography industry (a process that I take
to have been greatly accelerated by, but that predates, the internet).

I was reminded of the changing uses to which the word
\emph{pornographic} has been put when I happened to hear this exchange
regarding the Jodi Arias trial a couple of weeks ago, between host of
NPR's ``Talk of the Nation''
(\href{http://www.nytimes.com/2013/03/30/business/media/npr-to-end-talk-of-the-nation.html?_r=0}{requiescat
in pace}) Neal Conan and novelist Walter Mosley:

\begin{quote}
NEAL CONAN: \ldots{} is there a trial that you followed closely? WALTER
MOSLEY: You know, I actually try my best not to follow trials because
there seems to be something a little \textbf{pornographic} and a little
un-American about it. I kind of feel that if somebody is, like, being
tried for something, that that's - it's not exactly a private thing, but
it's a thing between them and the law, and that's the reason we have
law, so I don't have to make a decision about it.

(\href{http://www.lexisnexis.com/lnacui2api/api/version1/getDocCui?lni=58DH-PDP1-DY2S-N0JM\&csi=8398\&hl=t\&hv=t\&hnsd=f\&hns=t\&hgn=t\&oc=00240\&perma=true}{Why
We Can't Look Away from True-Life Courtroom Dramas}, May 13, 2013)
\end{quote}

What does ``pornographic'' here mean? In his comments, Mosley appeals to
a notion of privacy, as if the ``pornography'' of high-profile trials
(like that of Jodi Arias) inheres in a violation of the intimacy of a
special, private moment between the defendant and the court. {I will
leave entirely unremarked both the peculiar notion of privacy between an
individual and \emph{the state} and what may be \emph{un-American} about
it.}

But surely Mosley does not mean to indicate a violation of privacy, in
the strict sense, so much as the particular kind of spectatorship these
trials invite, the transformation of the trial from into a spectacle. A
trial, for such a perspective, is a sort of \emph{means} towards an
\emph{end} (call that end, say, justice); but the trial becomes a
spectacle when this means/ends logic is interrupted---when the end seems
to become obscured, and one revels in the trial as an end in itself. It
is a violation, not of any sort of privacy, so much as the usual
solemnity and function of trial.

This mode of spectatorship that confuses means and ends is what I take
Mosley to have meant when he called such trials \emph{pornographic}. Of
course, what counts as a means and what counts as an end are hardly
simple question; and this observation alone that doesn't \emph{really}
answer my broader question. Why is it that the term to name such
spectacle is \emph{pornographic}, rather than say \emph{sensational} or
\emph{exploitative} or even the bland and polemically boring
\emph{inappropriate}?

\hypertarget{food-porn}{%
\subsubsection{Food Porn}\label{food-porn}}

Petits pains au chocolat by Laurence Vagner, on Flickr

The best evidence of the shifting meanings of \emph{pornographic}, I
think, are the tumblrs, pinterest boards, and blogs that treat the term
\emph{porn} as nearly a suffix to name \emph{not} a particular kind of
content (though there are plenty of those) but rather the particular
mode of engagement that concerns Mosley when it is applied to trials.
Things like ``book porn,'' or ``library porn,'' or ``bookcase porn''
{With my examples, I fear I've given myself away.}. Surely, the most
common of such non-pornographic \emph{porn} (and likely the origin of
the instances I quoted and, no doubt, of many more) comes from
\emph{food porn}.

\emph{Food porn}, if you're unfamiliar, has its own
\href{http://en.wikipedia.org/wiki/Food_porn}{wikipedia entry}---though
that page betrays a variety (bordering on incoherent) of definitions
broader than what I would associate with the term {Among them, an
suggestion that the \emph{porn} of \emph{food porn} comes from its
nutritional qualities---i.e.~food porn is ``unhealthy'' food, as in this
\href{http://www.cspinet.org/nah/april98back.htm}{``Right Stuff vs.~Food
Porn'' eating advice column}.}. One can go to
\href{http://foodporndaily.com/}{Food Porn Daily} for what I take to be
instances of this particular genre. And while there's plenty of
variability in what gets tagged
\href{https://twitter.com/search?q=\%23foodporn}{\#foodporn}, the
instances at Food Porn Daily make the template pretty clear: pictures of
food with: highly saturated color; always in close up, frequently at a
low angle (a picture of cookies cooling on a baking sheet, shot from
what looks like just inches above the baking sheet); at least one area
of very crisp focus, often with a shallow depth of field (so that the
front of these
\href{http://foodporndaily.com/pictures/ricotta-crepes-with-smoked-salmon-capers-red-onions-and-cherry-tomatoes/}{ricotta
crepes with Smoked Salmon, Capers, Red Onions, and Cherry tomatoes} is
in sharp focus, even as the opposite end of the crepe blurs).

The food may be unplated, on a cutting board or still cooking in a pan;
it may be plated and presented, \emph{pret a manger}; occasionally one
finds a fork, having already separated out a bite, lying on the plate
ready to be picked up. Occasionally, one even sees a small bite already
taken (as is the case, I take it, in this image of a
\href{http://foodporndaily.com/pictures/prosciutto-strawberry-and-brie-panino-on-crusty-ciabatta/}{proscuitto,
strawberry, and brie panino on crusty ciabatta}. {\textbf{Banal
Hypothesis}: An analysis of the titles of menu items (à la Franco
Moretti's examination of the lengths of novel titles in ``Style Inc'')
would reveal a marked increase in complexity of dish names over the past
two decades.}. But any human presence is absent (though the image of a
half-eaten hamburger in
\href{http://www.huffingtonpost.com/2013/01/24/restaurant-photo-ban-death-of-food-porn_n_2543306.html}{this
story about the ``end of food porn''} might complicate that claim). The
photos at Food Porn Daily achieve a remarkable stylistic consistency
even though they are aggregated from a variety of sources. Indeed,
\href{http://en.wikipedia.org/wiki/Food_photography}{food photography}
appears to be a well established commercial specialization, and the
internet does not lack helpful
\href{http://digital-photography-school.com/food-photography-an-introduction}{introductions}
and
\href{http://articles.latimes.com/2013/apr/10/news/la-dd-food-photography-101-creating-food-porn-with-texture-20130409}{recommendations
for creating food porn}. That is, food porn has a well established set
of stylistic conventions, all of which work to solicit a certain type of
gaze. All the things that we might wish to enjoy in a piece of food are
translated (or perhaps merely replaced) by visual
analogues/correlates/substitutions/replacements. The papery texture of
crisply cooked pastry dough must be captured at the right depth, with
the right color, to suggest its materiality even in its absence; the
doneness of an egg yolk (warm not slimy, cooked through but not solid)
must inhere in the brightness of the yellow, perhaps even in the visual
evidence of the yoke's flow from the egg, cut only moments before by
some invisible hand.

If you wish to make these dishes, there is a
\href{http://www.amazon.com/Food-Porn-Daily-Amanda-Simpson/dp/1599553996/}{Food
Porn Daily cookbook}; though the amazon reviews confirm what surely you
already knew: these pictures, not the dishes, are themselves what is to
be consumed. You might try to make these dishes yourself; but really?
Expert though you may be, your dish will never look like that anyway
(even if you
\href{http://underthetuscangun.com/talk/foodography/an-inside-look-at-digital-food-photograhy/attachment/final-cereal-spoon-drip-1-of-1-2/}{replace
milk with Elmer's glue}). At one level this is no different than the
rise of the Food Network, or home improvement television, or other
genres of ``do it yourself'' entertainment which cloak entertainment and
aspiration in the ``how to'' and ``do it yourself'' rhetoric of
self-betterment.

But food porn (and its fellow pornographic genres) certainly intensifies
the divorce between means and ends. A bookshelf at
\href{http://bookshelfporn.com/}{bookshelfporn} is unlikely to inspire
one to remodel one's home, or take up saw and mallet. {Wood-grained
bookshelves and wood-grilled paninis? One could rewrite this entire post
focusing just on the issue of class. Look at these bookcases; look at
these dishes. Whose idea of the ``good life'' is this? And how do
adjectives and adjectival phrases (\emph{crunchy} cibatta bread, etc)
capture that aspiration?} And even if so inclined, you could never build
all the book shelves at bookshelfporn. Overabundance and excess are part
of the spectacle---one picture of a bookshelf is not bookshelfporn; but
once it's aggregated with others, you're on your way (perhaps a
bookshelf porn guide to carpentry is in the offing). In food porn, the
food itself is a necessary but not sufficient condition for the required
spectacle. Such photos represent a condition that in many cases never
existed---they are often postprocessed, or generated by techniques like
\href{http://deviantmonk.com/hdr-food-photography-tutorial/}{HDR}; the
macro lenses and lighting, the sort of focus and angle typical of these
photos creates a prosthetic, inhuman gaze.

So, here's a definitional distinction that I'll hazard. While Walter
Mosley can invoke the adjectival \emph{pornographic} to describe cable
news coverage of a trial, the term \emph{porn} (as a quasi-suffix to
indicate a certain genre of tumblr/blog/website) is reserved for visual
material, and indeed, for collections of photographic images. The
photographic conventions (which, it may be worth admitting, are not
unrelated to \emph{pornography} in the most conventional sense) are
easily transferable to nearly object that one could conceive a
\emph{lust} for (garden porn! bicycle porn! finch porn! banana peel
porn! linotype porn!){Please, please, please be careful about googling
any of those terms.}

Equally vital is the sense of an excessive superabudance of images
generated by the fact of aggregation and collection (or \emph{curation}
to speak in the argot of Web 2.0). Although food porn predates sites
like tumblr and pinboard, for such a definition it is through such sites
that (to adapt Hegelian terms) this particular idea of ``porn'' fully
achieves it Concept. The mode of spectatorship that is the defining fact
of this genre requires \emph{excess}. One finds that sort of excess
which itself inheres in the word \emph{obscene}:

\begin{quote}
\begin{enumerate}
\def\labelenumi{\arabic{enumi}.}
\setcounter{enumi}{1}
\tightlist
\item
  Offending against moral principles, repugnant; repulsive, foul,
  loathsome. Now (also): \emph{spec.} (of a price, sum of money, etc.)
  ridiculously or offensively high. \ldots{} 1974 Greenville (S.
  Carolina) News 23 Apr.~1/8 Energy officials have already predicted
  that first-quarter oil profits will be `embarrassingly high' or
  `whoppers'. Sen.~Henry Jackson, D-Wash., has said they'll be `almost
  obscene.'
\end{enumerate}

(``Obscene'')
\end{quote}

Another reason you can't eat the food porn: there is just too much of
it.

So, a definition. \textbf{Food porn}---a genre of ``\textbf{content}''
(and here too, using the vacuous term \emph{content}, the connection of
this genre to the history of the web seems evident) that redeploys
certain photographic conventions and a sense of excess in order to
solicit a mode of spectacular consumption, of consumption (of visual
apprehension) as an end in itself.

It is as ``an end in itself'' that it becomes possible, as Mosley does,
to use the term \emph{pornographic} to describe a way of watching a
trial \emph{and} to have the moral force of that statement be completely
clear. {He also has a not very convincing argument that contemporary
foodieism emerged out of the decline in sexual promiscuity in the wake
of the AIDS crisis.} When, in 2001 (a comparatively early moment in this
history of the idea of food porn), Anthony Bourdain invoked this same
sense of implicit morality, writing about ``food porn'' as a sort of
vicariousness (at times, he seems to cross the vehicle and tenor of the
metaphor, calling ``food porn'' not a substitute \emph{for food} but a
``substitute for sex'').

An obvious moral attends such a definition of imagery and the modes of
attention it solicits in terms of \emph{vicariousness}. Food porn
becomes a dangerous \emph{supplement} to real culinary enjoyment. {And,
indeed, D. H. Lawrence's objections to pornography of a more
conventional sort provides the template: pornography as a degraded,
perverted, alienated, vicarious sexuality (just as, from that wider
template of phonocentrism, writing is a sort of degraded/alienated
speech).} Bourdain, in 2001, expresses hope that we will soon move
beyond mere food porn to actual food:

\begin{quote}
As we once only read about sex in the '50s before indulging ourselves
indiscriminately in its pleasures in the '60s, '70s and early '80s, we
might now also be approaching a crossroads. Instead of simply reading
about small, good things and gaping at them in pictures, maybe we will
begin, once again after a long, long absence, to cook it, rediscovering
the best of ourselves and holding it close.

(\href{http://www.sfgate.com/books/article/ESSAY-FOOD-PORN-Lust-for-the-2862233.php\#ixzz2VFb3TzEM}{``Food
Porn''})
\end{quote}

{One must give Bourdain credit for mentions in the piece of the Olympia
Press, etc, which prove an admirable familiarity with the print history
of obscenity.} And, from a very different corner, one finds Frederic
Jameson drawing on this moral dimension of the term \emph{porn} in his
description of the pornographic in the opening lines from
\emph{Signatures of the Visible} (1990). Here though, it is not
\emph{vicariousness} (the replacement of the proper object by its mere
substitute) that is the problem, so much as the rapt, uncritical,
\emph{unthinking}, fascination which marks the pornographic gaze:

\begin{quote}
The visual is essentially pornographic, which is to say that it has its
end in rapt, mindless fascination; thinking about its attributes becomes
an adjunct to that, if it is unwilling to betray its object; while the
most austere films necessarily draw their energy from the attempt to
repress their own excess (rather than from the more thankless effort to
discipline the viewer). Pornographic films are thus only the
potentiation of films in general, which ask us to stare at the world as
though it were a naked body. (1)
\end{quote}

{The first sentence of that block quote won the \emph{Philosophy and
Literature}-sponsored third annual
\href{http://profron.net/fun/BadWriting.html}{``bad writing'' contest}.
I'm no fan of such contests, but I will grant that the demonstrative
that after the semicolon (which, I think, refers to ``that
fascination'') is a little ungainly. I wonder as well whether
\emph{potential} mightn't be a fairer term for \emph{potentiation}.} One
can find this formulation (``the visual is essentially pornographic'')
quoted with
\href{https://www.google.com/search?q=\%22visual+is+essentially+pornographic\%22\&btnG=Search+Books\&tbm=bks\&tbo=1}{some
frequency}. While it introduces \emph{Signatures of the Visible}, it is
not elaborated beyond this passage and serves chiefly as a way of
introducing a sort of statement of values and commitments that informs
Jameson's film criticism (of which \emph{Signatures} is a collection; I
recall liking the \emph{Dog Day Afternoon} essay).

And while Jameson says ``\emph{The visual} is essentially pornographic''
he (the thinker who offers Always historicize! as the ``slogan'' of
``all dialectical thought'' and the ``moral'' of \emph{The Political
Unconscious} {[}9{]}{Apparently, you can get a lot from a Jameson book
without ever moving past the first page.}), surely would not offer this
as a statement about ``the visual'' as such. Indeed, the thinker who
insists that ``the senses are themselves not natural organs but rather
the results of a long process of differentiation even within human
history'' (\emph{Political} 62), is making not a statement about the
visual \emph{as such}, but about the visual at our own historical
moment---or rather, the place of the visual at the start of 1990s, prior
to what we might imagine as our own post-cinematic moment.

Indeed, there are two histories that \emph{The Political Unconscious}
offers that help account for the suffixication of ``porn'' to name a
genre of browsable imagery defined \emph{not} by the content of the
images, but by a mode of engagement (``rapt fascination'').

Discussing psychoanalysis, Jameson suggests that the emergence of the
psychoanalytic hermeneutic depends on a larger autonomization of
sexuality itself:

\begin{quote}
The psychoanalytic demonstration of the sexual dimension of overtly
nonsexual conscious experience and behavior is possible only when the
sexual ``dispositif'' or apparatus has by a process of isolation,
autonomization, specialization, developed into an independent sign
system or symbolic dimension in its own right; as long as sexuality
remains as integrated into social life in general as say, eating, its
possibilities of symbolic extension are to that degree limited, and the
sexual retains its status as a banal inner-worldly event and bodily
function. (\emph{Political} 64)
\end{quote}

Food porn, like psychoanalysis, reveals the ``sexual dimension of the
overtly nonsexual'' as well. Not, however, the same way psychoanalysis
does---through a hermeneutics of revealing, discovering a latent
sexuality behind some overt, apparently nonsexual meanings. The
apparatus at stake here is the pornographic gaze itself, which has
itself been sufficiently autonomized to be utterly separable even from
pornography.

There is a second, broader, history out of which food porn emerges: that
of aesthetics itself. Insomuch as food porn (and similar phenomena)
represents the consuming of images as images, as a quasi-autonomous
experience that is only tangentially connected to the culinary
enjoyment{Adorno refers at
\href{http://books.google.com/books?id=Wk_NFsGTkncC\&lpg=PA212\&dq=\%22merely\%20culinary\%22\%20inauthor\%3Aadorno\&pg=PA212\#v=onepage\&q=\%22merely\%20culinary\%22\%20inauthor:adorno\&f=false}{one
point} to non-aesthetic pleasure as \emph{merely culinary}.} (or
whatever) that they appear to represent. Of the emergence of visual art
from the autonomization of sight itself, Jameson writes:

\begin{quote}
as sight becomes a separate activity in its own right, it acquires new
objects that are themselves the products of a process of abstraction and
rationalization which strips the experience of the concrete of such
attributes as color, spatial depth, texture, and the like, which in
their turn undergo reification. The history of forms evidently reflects
this process, by which the visual features of ritual, or those practices
of imagery still functional in religious ceremonies, are secularized and
reorganized into ends in themselves, in easel painting and new genres
like landscape, then more openly in the perceptual revolution of the
impressionists, with the autonomy of the visual finally triumphantly
proclaimed in abstract expressionism. So Lukács is not wrong to
associate the emergence of modernism with the reification which is its
precondition; but he oversimplifies and deproblematizes a complicated
and interesting situation by ignoring the Utopian vocation of the newly
reified sense, the mission of this heightened and autonomous language of
color to restore at least symbolic experience of libidinal gratification
to a world drained of it, a world of extension, gray and merely
quantifiable. (\emph{Political} 63)
\end{quote}

The engagement solicited by food porn, and indeed by \emph{porn}, is
itself not unrelated to the aesthetic gaze and its history which Jameson
offers here in miniature. The aesthetic as a separate domain insists on
such autonomy. In a provocative aside in her essay on ``Jane Austen and
the Masturbating Girl,'' Eve Sedgwick connects the autonomous aesthetic
of Kant to autoerotic pleasure: ``the Aesthetic in Kant is both
substantially indistinguishable from, and at the same time
definitionally opposed against, autoerotic pleasure'' (111).
\emph{Definitionally opposed} because, in the Kantian vocabulary, the
pornographic would be \emph{merely agreeable}; it is interested pleasure
which gratifies some bodily need. The pleasure one takes in eating an
hamburger, for Kant, is not aesthetic because it sates a hunger. But if
I take pleasure not in the hamburger, but in an image of it, my pleasure
begins to look less interested, less purposeful even if it maintains a
certain\ldots{} purposiveness; it begins to look, maybe eerily,
aesthetic.

Of course, this slippery slope which leads to the equation of
\emph{porn} with the aesthetic is one which the twentieth-century
assiduously avoided, even as the substantial indistinguishability
Sedgwick notes persisted. One can see it, for instance, in the
similarity of the formulae \emph{art for art's sake} and \emph{dirt for
dirt's sake}. This latter formula was invoked by Judge Woolsey in the
famous 1933 \emph{Ulysses} decision (one finds it today reprinted with
the Viking edition). In finding that the work was nowhere ``dirt for
dirt's sake,'' Woolsey finds that the work is not obscene. And yet, that
slogan itself, which offers a definition of obscenity, deliberately
echoes the slogan-like assertion of aesthetic autonomy, ``art for art's
sake.''

\hypertarget{works-cited}{%
\subsection{Works Cited}\label{works-cited}}

\begin{itemize}
\item
  Jameson, Frederic. \emph{The Political Unconscious}. Cornell, NY:
  Cornell UP, 1981. Print.
\item
  ---------. \emph{Signatures of the Visible}. New York: Routledge,
  1990. Print.
\item
  ``obscene, adj.'' OED Online. March 2013. Oxford University Press.
  Online.
\item
  Sedgwick, Eve Kosofsky. \emph{Tendencies}. Durham, Duke UP, 1993.
  Print.
\end{itemize}

\end{document}
