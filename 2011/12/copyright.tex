% Options for packages loaded elsewhere
\PassOptionsToPackage{unicode}{hyperref}
\PassOptionsToPackage{hyphens}{url}
%
\documentclass[
  12pt,
]{article}
\usepackage{lmodern}
\usepackage{amssymb,amsmath}
\usepackage{graphicx}
\usepackage{ifxetex,ifluatex}
\ifnum 0\ifxetex 1\fi\ifluatex 1\fi=0 % if pdftex
  \usepackage[T1]{fontenc}
  \usepackage[utf8]{inputenc}
  \usepackage{textcomp} % provide euro and other symbols
\else % if luatex or xetex
  \usepackage{unicode-math}
  \defaultfontfeatures{Scale=MatchLowercase}
  \defaultfontfeatures[\rmfamily]{Ligatures=TeX,Scale=1}
  \setmainfont[]{Hoefler Text}
  \setsansfont[]{Gill Sans}
\fi
% Use upquote if available, for straight quotes in verbatim environments
\IfFileExists{upquote.sty}{\usepackage{upquote}}{}
\IfFileExists{microtype.sty}{% use microtype if available
  \usepackage[]{microtype}
  \UseMicrotypeSet[protrusion]{basicmath} % disable protrusion for tt fonts
}{}
\makeatletter
\@ifundefined{KOMAClassName}{% if non-KOMA class
  \IfFileExists{parskip.sty}{%
    \usepackage{parskip}
  }{% else
    \setlength{\parindent}{0pt}
    \setlength{\parskip}{6pt plus 2pt minus 1pt}}
}{% if KOMA class
  \KOMAoptions{parskip=half}}
\makeatother
\usepackage{xcolor}
\IfFileExists{xurl.sty}{\usepackage{xurl}}{} % add URL line breaks if available
\IfFileExists{bookmark.sty}{\usepackage{bookmark}}{\usepackage{hyperref}}
\hypersetup{
  pdftitle={English Department Undergraduate Courses},
  hidelinks,
  pdfcreator={LaTeX via pandoc}}
\urlstyle{same} % disable monospaced font for URLs
\usepackage[paperheight=8.5in,paperwidth=5.5in,bottom=0.75in,top=0.75in,left=0.5in,right=0.5in]{geometry}
\setlength{\emergencystretch}{3em} % prevent overfull lines
\providecommand{\tightlist}{%
  \setlength{\itemsep}{0pt}\setlength{\parskip}{0pt}}
\setcounter{secnumdepth}{-\maxdimen} % remove section numbering
\ifluatex
  \usepackage{selnolig}  % disable illegal ligatures
\fi


\newlength{\cslhangindent}
\setlength{\cslhangindent}{1.5em}
\newlength{\csllabelwidth}
\setlength{\csllabelwidth}{3em}
\newlength{\cslentryspacingunit} % times entry-spacing
\setlength{\cslentryspacingunit}{\parskip}
\newenvironment{CSLReferences}[2] % #1 hanging-ident, #2 entry spacing
 {% don't indent paragraphs
  \setlength{\parindent}{0pt}
  % turn on hanging indent if param 1 is 1
  \ifodd #1
  \let\oldpar\par
  \def\par{\hangindent=\cslhangindent\oldpar}
  \fi
  % set entry spacing
  \setlength{\parskip}{#2\cslentryspacingunit}
 }%
 {}
\usepackage{calc}
\newcommand{\CSLBlock}[1]{#1\hfill\break}
\newcommand{\CSLLeftMargin}[1]{\parbox[t]{\csllabelwidth}{#1}}
\newcommand{\CSLRightInline}[1]{\parbox[t]{\linewidth - \csllabelwidth}{#1}\break}
\newcommand{\CSLIndent}[1]{\hspace{\cslhangindent}#1}

\usepackage{multicol}


\title{The Changing Index of Censorship}
\author{}
\date{}

\begin{document}

(\textbf{Jan.~17 Update:} In a truly remarkable show of solidarity, as
you dear reader have likely noticed by now, major sites (including
wikipedia) across the internet are mainting their protest of SOPA on
January 18th. I've add \href{http://sopablackout.org/}{this bit of
javascript} to show solidarity with them. Despite the shelving of the
house bill, these sites all recognize the continued threat of SOPA,
PIPA, or any similar legislation.)

(\textbf{Jan.~16 Update:} With the news that, at least for now,
\href{http://boingboing.net/2012/01/16/sopa-is-dead-its-evil-senate.html}{SOPA
is shelved}, I've removed the javascript mentioned below.)

Depending on when you visit this page, you may notice that a whole bunch
of it is blacked out; that's a result of
\href{https://github.com/dougmartin/Stop-Censorship}{this bit} of
javascript, protesting
\href{http://en.wikipedia.org/wiki/Stop_Online_Piracy_Act}{SOPA}. If you
haven't heard about SOPA,
\href{http://www.theverge.com/2011/12/22/2648219/stop-online-piracy-act-sopa-what-is-it}{here}
is a fine place to start.

As someone with an interest in the history of obscenity and censorship I
have been impressed at how quickly groups like the
\href{http://eff.org}{EFF} have described SOPA as a matter of censorship
and free speech. Unlike the 1996 Communications Decency Act (I recall
the blue ribbon GIFs it inspired well), the object of SOPA is not online
obscenity. As these groups recognize, however, matters of ``free
speech'' are increasingly questions of intellectual property and the
technologies of copyright enforcement.

This development is entirely consistent with the narrative offered by
one of my favorite early-twentieth century anti-censorship tracts,
\emph{To the Pure: A Study of Obscenity and the Censor}, co-authored by
William Seagle and Morris Ernst (the lawyer who, a few years later,
defended \emph{Ulysses} in front of Judge John Woolsey). \emph{To The
Pure} outlines a political history of censorship which proceeds through
three stages; in a section titled ``The Changing Index of Censorship''
Ernst and Seagle explain:

\begin{quote}
With the invention of printing in the middle of the fifteenth century,
the first condition for the censorship of literature began to be
fulfilled: literature was on its way to popular distribution. The three
forms of censorship which we know today began to develop: (1) the
religious (2) the political (3) the sexual, which is the modern
culmination. The course of evolution may be stated to be from heresy to
treason to obscenity. The purpose of authority remains always the same,
but the index of censorship changes. Each age produces those formulae of
suppression which coincide with its dominant interest. (140)
\end{quote}

\textbf{Each age produces those formulae of suppression which coincide
with its dominant interest}. The three chief categories of prohibited
speech---blasphemy, sedition, obscenity---reflect a historical
trajectory. Despite its obvious oversimplifications, Ernst and Seagle's
account of this progression surprises me with its insight whenever I
return to it.

If one were to continue the history of censorship Ernst began nearly a
century ago, it would pass through questions of intellectual property.
Rather than obscenity, it is \emph{piracy} which is the object of
contemporary censorship just as obscenity and blasphemy were the objects
of a previous age's censorship regime.

We certainly shouldn't ignore other, more traditional, types of
censorship which continue to exist (particularly outside of the United
States and western Europe); however, if \emph{blasphemy} was the object
of censorship in (what Ernst calls) the ``Age of Faith,''
\emph{sedition} in the ``Age of Divine Right,'' and \emph{obscenity} in
the ``Age of Democracy,'' piracy is the crime that the censorship
regimes of our own information age seek to control. The \emph{Index
Librorum Prohibitorum} and Customs lists of previous eras bear a
striking resemblance to the index of prohibited foriegn domains that
SOPA would create.

\hypertarget{works-cited}{%
\subsection{Works Cited}\label{works-cited}}

\begin{itemize}
\tightlist
\item
  Ernst, Morris and William Seagle. \emph{To The Pure: A Study of
  Obscenity and the Censor}. New York: Viking Press, 1928. Print.
\end{itemize}

\end{document}
