% Options for packages loaded elsewhere
\PassOptionsToPackage{unicode}{hyperref}
\PassOptionsToPackage{hyphens}{url}
%
\documentclass[
  12pt,
]{article}
\usepackage{lmodern}
\usepackage{amssymb,amsmath}
\usepackage{graphicx}
\usepackage{ifxetex,ifluatex}
\ifnum 0\ifxetex 1\fi\ifluatex 1\fi=0 % if pdftex
  \usepackage[T1]{fontenc}
  \usepackage[utf8]{inputenc}
  \usepackage{textcomp} % provide euro and other symbols
\else % if luatex or xetex
  \usepackage{unicode-math}
  \defaultfontfeatures{Scale=MatchLowercase}
  \defaultfontfeatures[\rmfamily]{Ligatures=TeX,Scale=1}
  \setmainfont[]{Hoefler Text}
  \setsansfont[]{Gill Sans}
\fi
% Use upquote if available, for straight quotes in verbatim environments
\IfFileExists{upquote.sty}{\usepackage{upquote}}{}
\IfFileExists{microtype.sty}{% use microtype if available
  \usepackage[]{microtype}
  \UseMicrotypeSet[protrusion]{basicmath} % disable protrusion for tt fonts
}{}
\makeatletter
\@ifundefined{KOMAClassName}{% if non-KOMA class
  \IfFileExists{parskip.sty}{%
    \usepackage{parskip}
  }{% else
    \setlength{\parindent}{0pt}
    \setlength{\parskip}{6pt plus 2pt minus 1pt}}
}{% if KOMA class
  \KOMAoptions{parskip=half}}
\makeatother
\usepackage{xcolor}
\IfFileExists{xurl.sty}{\usepackage{xurl}}{} % add URL line breaks if available
\IfFileExists{bookmark.sty}{\usepackage{bookmark}}{\usepackage{hyperref}}
\hypersetup{
  pdftitle={English Department Undergraduate Courses},
  hidelinks,
  pdfcreator={LaTeX via pandoc}}
\urlstyle{same} % disable monospaced font for URLs
\usepackage[paperheight=8.5in,paperwidth=5.5in,bottom=0.75in,top=0.75in,left=0.5in,right=0.5in]{geometry}
\setlength{\emergencystretch}{3em} % prevent overfull lines
\providecommand{\tightlist}{%
  \setlength{\itemsep}{0pt}\setlength{\parskip}{0pt}}
\setcounter{secnumdepth}{-\maxdimen} % remove section numbering
\ifluatex
  \usepackage{selnolig}  % disable illegal ligatures
\fi


\newlength{\cslhangindent}
\setlength{\cslhangindent}{1.5em}
\newlength{\csllabelwidth}
\setlength{\csllabelwidth}{3em}
\newlength{\cslentryspacingunit} % times entry-spacing
\setlength{\cslentryspacingunit}{\parskip}
\newenvironment{CSLReferences}[2] % #1 hanging-ident, #2 entry spacing
 {% don't indent paragraphs
  \setlength{\parindent}{0pt}
  % turn on hanging indent if param 1 is 1
  \ifodd #1
  \let\oldpar\par
  \def\par{\hangindent=\cslhangindent\oldpar}
  \fi
  % set entry spacing
  \setlength{\parskip}{#2\cslentryspacingunit}
 }%
 {}
\usepackage{calc}
\newcommand{\CSLBlock}[1]{#1\hfill\break}
\newcommand{\CSLLeftMargin}[1]{\parbox[t]{\csllabelwidth}{#1}}
\newcommand{\CSLRightInline}[1]{\parbox[t]{\linewidth - \csllabelwidth}{#1}\break}
\newcommand{\CSLIndent}[1]{\hspace{\cslhangindent}#1}

\usepackage{multicol}


\title{About 1400 Words of Skepticism about Markdown, and an Imagined
Alternative}
\author{}
\date{2015-06-29}

\begin{document}

Don't get me wrong,
\href{http://daringfireball.net/projects/markdown/}{Markdown}'s great.
Indeed, nearly all the writing I do now is in Markdown (or at least
starts that way). There has been a good amount of writing about the
virtues of Markdown for academic writing in particular, so I'll just
link to them here:

\begin{itemize}
\tightlist
\item
  W. Caleb McDaniel's
  \href{http://wcm1.web.rice.edu/my-academic-book-in-plain-text.html}{Why
  (and How) I Wrote My Academic Book in Plain Text}
\item
  Nikola Sander's
  \href{http://nikolasander.com/writing-in-markdown/}{Writing Academic
  Papers in Markdown Using Sublime Text and Pandoc}
\item
  Dennis Tennen and Grant Wythoff's
  \href{http://programminghistorian.org/lessons/sustainable-authorship-in-plain-text-using-pandoc-and-markdown}{Sustainable
  Authorship in Plaintext Using Pandoc and Markdown} (This latter I
  especially recommend).
\end{itemize}

But Markdown, as it stands, has some drawbacks, which become acute when
you are trying to extend it to cover the needs of academic writing (or,
say, as \href{http://web.uvic.ca/~mvp1922/otsummit/}{a transcription
format for texts}).

\hypertarget{the-problem}{%
\subsection{The Problem}\label{the-problem}}

What I will describe as ``problems'' all stem from the fact that
Markdown remains essentially a simplified syntax for HTML. A tool like
\href{http://pandoc.org}{Pandoc}, which has a special (and especially
powerful) flavor of Markdown all its own, helps reduce the borders
between document formats. With Pandoc it becomes easy to convert
\texttt{HTML} to \texttt{LaTeX}, or Rich Text Format to Word's
\texttt{.docx}. It could easily feel like Markdown is a universal
document format---write it in Markdown, and publish as whatever.

That is a lovely dream---an easy-to-write plaintext format that can
easily be output to any desired format. In reality, though, Markdown
(even Pandoc's Markdown) remains yoked to HTML, and so it suffers from
some of its problems.

The problem I encounter most frequently in HTML (and in Markdown)
concerns nesting a block quote within a paragraph. In short, can you
have a block quote \emph{within} a paragraph? If you're writing HTML (or
MarkDown), the answer is no---HTML treats ``block quotes'' as
\texttt{block} \emph{elements}; this means that one cannot be contained
within a paragraph (this restriction does not exist in LaTeX or TEI).
Yet, what could be more common in writing on works of literature?
Representing poetry presents its own problems for HTML and Markdown.{By
contrast to the challenge presented by the mere fact of poetry, note the
many syntaxes/tools available for fenced code blocks, syntax
highlighting, and so on; Markdown, for now, remains of greatest interest
to software developers and so reflects their habits and
needs.}(\emph{Note}: If you're looking for practical advice, you can
easily represent poetry in Pandoc's markdown using
\href{http://pandoc.org/README.html\#line-blocks}{``line blocks''}; this
is not a perfect solution, but it will do for many needs).

Perversely, markdown also represents something of a step backward with
regard to \emph{semantics}. If you've spent some time with HTML, you may
have noticed how HTML5 cements a model of HTML as a semantic markup
language (with, implicitly, matters of presentation controlled by CSS).
That means that the \texttt{\textless{}i\textgreater{}} tag, which long
ago meant \emph{italics}, has since acquired semantic meaning.
\href{https://docs.webplatform.org/wiki/html/elements/i}{According to
the w3c}, it should be used to ``represent{[}{]} a span of text offset
from its surrounding content without conveying any extra emphasis or
importance, and for which the conventional typographic presentation is
italic text; for example, a taxonomic designation, a technical term, an
idiomatic phrase from another language, a thought, or a ship name.''
Those instances where one wishes to express emphasis, use the
\texttt{\textless{}em\textgreater{}} tag. If you need to mark a title,
don't simply italicize it, use
\href{https://docs.webplatform.org/wiki/html/elements/cite}{\texttt{\textless{}cite\textgreater{}}}
.{But hold up, that \texttt{cite} element obscures the distinctions we
normally make between italicizing certain titles and putting others in
quotation marks.} In practice, of course, I doubt these distinctions are
widely respected across the web; but all those at least
\emph{potentially} useful distinctions are lost in markdown, whose
syntax marks them all with \texttt{*} or \texttt{\_}. Markdown is, in
fact, rather \emph{unsemantic}. (To a lesser degree, one might detect
this tendency as well in the way headings---rather than
\texttt{divs}---are Markdown's primary way of structuring a document,
but I'll stop now.) So, two points: Markdown inherits HTML's document
which includes an inability to nest block-level elements within
paragraphs; in simplifying HTML, it produces a less semantically clear
and rich format. (Technically, of course, one could simply include any
HTML element for which Markdown offers no shortened syntax---like
\texttt{\textless{}cite\textgreater{}} for example.)

\hypertarget{a-solution}{%
\subsection{A Solution}\label{a-solution}}

On the \href{http://talk.commonmark.org/}{CommonMark forum}, some folks
have proposed additional syntax to fix the latter problem, and capture
some of the semantic distinctions mentioned above (indeed, following the
discussions over there has helped sensitize to me some of the challenges
and limitations of markdown as a sort of universal format donor). So,
some of these issues could be resolved through extensions or
modifications of Markdown.

Yet, given these deficits in Markdown, I wonder if it isn't worth asking
a more basic question---whether the plaintext format for ``academic''
writing should be so tightly yoked to HTML? If Markdown is,
fundamentally, a simplified, plaintext syntax for HTML, could we imagine
a similar, easy-to-write, plaintext format that wouldn't be tied to
HTML? Could we imagine, say, a format that would represent a
simplification of syntax, not of HTML, but of a format better suited to
the needs of representing more complex documents? Could we imagine a
plaintext format that would be to
\href{http://www.tei-c.org/index.xml}{TEI}, say, what markdown is to
HTML?

Such a format would not need to \emph{look} particularly different from
Markdown. Its syntax could overlap significantly; as in Pandoc's
Markdown format, file metadata (things like title, author, and so on)
could appear (perhaps as YAML) at the front of the file (and be
converted into elements within \texttt{teiHeader}). You could still use
\texttt{*}, \texttt{**}, and \texttt{{[}{]}()} as your chief tools;
footnotes and references could be marked the same way (you could
preserve Pandoc's wonderful citation system, with such things
represented as \texttt{\textless{}refs\textgreater{}} in TEI).

The most substantive difference would not be in syntax, but in the
document model. Any Markdown file can contain HTML---all HTML is valid
markdown; this ensures that Markdown is never less powerful than HTML.
But are the burdens of HTML worth the costs if one wishes to do
scholarly/academic, or similar types of writing, in plaintext? Projects
exist to repurpose Pandoc markdown for scholarly writing: Tim T. Y.
Lin's
\href{http://scholarlymarkdown.com/Scholarly-Markdown-Guide.html\#first-steps}{ScholarlyMarkdown},
or Martin Fenner's
\href{http://blog.martinfenner.org/2013/06/29/metadata-in-scholarly-markdown/}{similar
project}, or the workflow linked-to above, by Dennis Tennen and Grant
Wythoff at the
\href{http://programminghistorian.org/lessons/sustainable-authorship-in-plain-text-using-pandoc-and-markdown}{Programming
Historian}. What I'm imagining, though, is entirely less practical than
any of these projects at the moment because it would necessitate a
change in the document model into which markdown is converted. Pandoc
works its magic by reading documents from a source format (through a
``reader'') into an intermediary format (a format of its own that you
can view by outputting \texttt{-t\ native}), which it can then output
(through a ``writer''). Could TEI (or some representation of it),
essentially, fulfill that role as intermediary format? (A Pandoc car
with a TEI engine swapped in?)

I like writing in plaintext, but I don't love being bound by the
peculiarities that Markdown has inherited from HTML. So, it is worth
considering what it is that people like about Markdown. I suspect that
most of the things people like about Markdown (free, easy to write,
nonproprietary, easily usable with version control, and so on), have
little to do with its HTML-based document model but stem from its being
a plaintext format (and the existing infrastructure of
scripts/apps/workflows around markdown). TEI provides an alternative
document model---indeed, a \emph{richer} document model. Imagine a
version of Pandoc that uses TEI (or a simplified TEI subset) behind the
scenes as its native format. Folks often complain about the complexity
and verbosity of TEI (and XML more generally), and not without reason. I
would certainly never want to \emph{write} TEI; but a simplified TEI
syntax that could then take advantage of all the virtues of TEI, that
would be something.

{[}Closing Note: At one point I wondered how easy it would be to convert
markdown to TEI with Pandoc\ldots{} I've managed to finagle a set of
scripts to do that; it's janky, but for anyone interested, it's
\href{https://github.com/c-forster/markdown2tei}{here}.{]}

\end{document}
