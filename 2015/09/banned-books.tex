% Options for packages loaded elsewhere
\PassOptionsToPackage{unicode}{hyperref}
\PassOptionsToPackage{hyphens}{url}
%
\documentclass[
  12pt,
]{article}
\usepackage{lmodern}
\usepackage{amssymb,amsmath}
\usepackage{graphicx}
\usepackage{ifxetex,ifluatex}
\ifnum 0\ifxetex 1\fi\ifluatex 1\fi=0 % if pdftex
  \usepackage[T1]{fontenc}
  \usepackage[utf8]{inputenc}
  \usepackage{textcomp} % provide euro and other symbols
\else % if luatex or xetex
  \usepackage{unicode-math}
  \defaultfontfeatures{Scale=MatchLowercase}
  \defaultfontfeatures[\rmfamily]{Ligatures=TeX,Scale=1}
  \setmainfont[]{Hoefler Text}
  \setsansfont[]{Gill Sans}
\fi
% Use upquote if available, for straight quotes in verbatim environments
\IfFileExists{upquote.sty}{\usepackage{upquote}}{}
\IfFileExists{microtype.sty}{% use microtype if available
  \usepackage[]{microtype}
  \UseMicrotypeSet[protrusion]{basicmath} % disable protrusion for tt fonts
}{}
\makeatletter
\@ifundefined{KOMAClassName}{% if non-KOMA class
  \IfFileExists{parskip.sty}{%
    \usepackage{parskip}
  }{% else
    \setlength{\parindent}{0pt}
    \setlength{\parskip}{6pt plus 2pt minus 1pt}}
}{% if KOMA class
  \KOMAoptions{parskip=half}}
\makeatother
\usepackage{xcolor}
\IfFileExists{xurl.sty}{\usepackage{xurl}}{} % add URL line breaks if available
\IfFileExists{bookmark.sty}{\usepackage{bookmark}}{\usepackage{hyperref}}
\hypersetup{
  pdftitle={English Department Undergraduate Courses},
  hidelinks,
  pdfcreator={LaTeX via pandoc}}
\urlstyle{same} % disable monospaced font for URLs
\usepackage[paperheight=8.5in,paperwidth=5.5in,bottom=0.75in,top=0.75in,left=0.5in,right=0.5in]{geometry}
\setlength{\emergencystretch}{3em} % prevent overfull lines
\providecommand{\tightlist}{%
  \setlength{\itemsep}{0pt}\setlength{\parskip}{0pt}}
\setcounter{secnumdepth}{-\maxdimen} % remove section numbering
\ifluatex
  \usepackage{selnolig}  % disable illegal ligatures
\fi


\newlength{\cslhangindent}
\setlength{\cslhangindent}{1.5em}
\newlength{\csllabelwidth}
\setlength{\csllabelwidth}{3em}
\newlength{\cslentryspacingunit} % times entry-spacing
\setlength{\cslentryspacingunit}{\parskip}
\newenvironment{CSLReferences}[2] % #1 hanging-ident, #2 entry spacing
 {% don't indent paragraphs
  \setlength{\parindent}{0pt}
  % turn on hanging indent if param 1 is 1
  \ifodd #1
  \let\oldpar\par
  \def\par{\hangindent=\cslhangindent\oldpar}
  \fi
  % set entry spacing
  \setlength{\parskip}{#2\cslentryspacingunit}
 }%
 {}
\usepackage{calc}
\newcommand{\CSLBlock}[1]{#1\hfill\break}
\newcommand{\CSLLeftMargin}[1]{\parbox[t]{\csllabelwidth}{#1}}
\newcommand{\CSLRightInline}[1]{\parbox[t]{\linewidth - \csllabelwidth}{#1}\break}
\newcommand{\CSLIndent}[1]{\hspace{\cslhangindent}#1}

\usepackage{multicol}


\title{Two Cheers for Banned Books Week}
\author{}
\date{2015-09-30}

\begin{document}

{My title, recalling E. M. Forster's (no relation, sadly) \emph{Two
Cheers for Democracy}, might be too generous. I can't imagine mustering
more than two cheers for anything. Two is probably the utter limit of my
cheering.}

Is ``Banned Books Week'' anarchronistic? That's the claim of
\href{http://www.slate.com/articles/arts/culturebox/2015/09/banned_books_week_no_one_bans_books_anymore_and_censorship_of_books_is_incredibly.single.html}{this
article at Slate}. John Overholt, in
\href{https://twitter.com/john_overholt/status/648471105204830208}{a
single tweet}, manages to voice what I think are all the most pressing
complaints about such a perspective:

A. "We won" is a huge oversimplification.
B. It\textquotesingle s only true for the narrowest definition of ban.
C. They keep trying. http://t.co/iHo3BNbGx9

--- John Overholt ((\textbf{john\_overholt?})) September 28, 2015

And yet, as someone who spends quite a bit of time trying to think about
what early-twentieth century censorship means, there is an important
grain of truth in the Slate piece that is worth preserving, even if
declarations of ``Mission Accomplished'' feel premature.

The idea of book banning conjures images of state censorship and book
burning---images one can find in abundance, for instance, in Kevin
Birmingham's wonderful account \emph{Ulysses},
\href{http://www.nybooks.com/articles/archives/2015/apr/23/ulysses-its-still-scandal/}{\emph{The
Most Dangerous Book}}; images of court rooms where lawyers make grand
appeals to \emph{literary value} and \emph{freedom of expression}.{I
pass over in silence distinctions between \emph{books} and
\emph{literature}\ldots{}} The scene that Michelle Anne Schingler
describes, at
\href{http://bookriot.com/2015/09/29/hey-slate-banned-books-week-isnt-crock/}{Book
Riot}, however is a very different one. Schingler wastes no time in
affirming ``No,'' book banning is not simply over. Her examples all
concern libraries; and indeed as Graham contends in her piece at Slate,
the most recent cases that the
\href{http://www.bannedbooksweek.org/about}{Banned Books website} cites
all concern attempts to limit or remove books from library collections
or school settings.

When Graham declares ``we won,'' that ``book banning'' is over, this is
indeed (as Overholt suggests) an oversimplification. It imagines a
single struggle, which reaches some sort of crisis, and ends. Graham is
offering a narrative very close to that recounted by Charles Rembar in
his memoir \emph{The End of Obscenity}. Rembar was the defense attorney
during many of the key 1960s obscenity trials in the United States, and
his memoir wonderfully charts the erosion of state censorship in the
period. Suppression of books on grounds of obscenity, Rembar suggests (I
think, rightly), ends after the trials of \emph{Lady Chatterley's
Lover}, \emph{Tropic of Cancer}, and \emph{Fanny Hill} in the United
States. Starting in
\href{http://www.oyez.org/cases/1950-1959/1956/1956_582}{Roth v. United
States}, and culminating ultimately in the so-called
\href{http://en.wikipedia.org/wiki/Miller_test}{``Miller Test''},
American jurisprudence evolves a set of standards that have the effect
of ending the censorship of books on the grounds of obscenity.{In
English jurisprudence, the 1959 reform of the Obscene Publications Act
(which enabled publication of Lawrence's \emph{Lady Chatterley} by
Penguin) plays the same role as the court cases discussed by Rembar.}
After those trials, it has proved essentially impossible for a book to
be banned on grounds of obscenity; contract, libel, and copyright all
continue to shape cultural production in important ways (the latter
especially so), but obscenity and its particular brand of
state-controlled book burning is indeed over. The Miller standard may
justifiably be celebrated as a sort of liberal triumph.

\par\begin{figure}\centering\includegraphics[width=\columnwidth]{_images/miller_logo_29850.jpg}\caption{Miller Logo}\end{figure}

And boy, do we love to tell this story. Birmingham offers a version in
his account of \emph{Ulysses}; we get a sort of version in
\href{http://www.imdb.com/title/tt1049402/}{movies about Allen
Ginsberg's ``Howl''}; or in TV movies about
\href{http://www.imdb.com/title/tt0757175/}{the \emph{Chatterley}
trial}. Folks love this tale of heroic lawyers fighting on behalf of
great works of literature, against philistine puritans---figures like
Anthony Comstock or William Joynson-Hicks (more commonly called simply,
``Jix''). We tell very similar stories about Elvis and his hips, or
Lenny Bruce and his comedy---tales where transgression and freedom
contend with (usually comicly absurd) conservatism. (We even tell a
version of this story about
\href{https://en.wikipedia.org/wiki/Footloose_(1984_film)}{dancing in
small towns}.) It's usually a narrative of triumph, told by liberal
proponents who indeed end by declaring ``We won.'' And, as history, it
is usually an oversimplication.{For one thing (and this is a hobby horse
of mine), it tends to remove books from a broader media history which
shapes what it means to ``ban'' a ``book.'' I have a different story of
my own, about the place of literature in the changing media ecology of
the long twentieth century\ldots{} but that's another tale for another
time.}

\par\begin{figure}\centering\includegraphics[width=\columnwidth]{_images/FootloosePoster.jpg}\caption{Footloose}\end{figure}

``Banned Books Week'' conflates two narratives, perhaps deliberately. It
inserts present instances of political struggle which involve books,
particularly (like those described by Schingler) around libraries, into
a longer history of book banning. It is, in some ways, a savvy
rhetorical move to align parents who want to limit access to particular
titles with Anthony Comstock and similar figures (after all, who wants
to be \href{https://www.youtube.com/watch?v=20jbY6awlTw}{this guy}). But
this conflation also has the, I think unfortunate, effect of casting
contemporary debates about education and the meaning of ``the public''
as matters of ``banning books.'' I think it makes more sense to
understand
\href{http://bannedbooks.world.edu/2012/01/22/banned-books-awareness-beloved/}{attempts
to limit access to Toni Morrison's \emph{Beloved}}, not as a debate
about book censorship continuous with the suppression of \emph{Ulysses}
or \emph{Lady Chatterley's Lover}, but as part of the same political
struggle over
\href{https://www.washingtonpost.com/opinions/whitewashing-civil-war-history-for-young-minds/2015/07/06/1168226c-2415-11e5-b77f-eb13a215f593_story.html}{how
to teach the causes of the American Civil War}, or even
\href{http://www.nytimes.com/2013/11/23/education/texas-education-board-flags-biology-textbook-over-evolution-concerns.html}{whether
to mention evolution}. These are debates about books; but more
fundamentally they are debates about education and, more importantly,
debates about \emph{the public}. They ask not, ``Should this book be
legally available?'', but ``Should \emph{my children} learn this?'' or
``Should \emph{my tax dollars} pay for this?'' Defending against the
active defunding of public goods by appealing to the ``freedom to
read,'' seems to me, to be a tactic of ambivalent value.

When ``Banned Books Week'' began in 1982, the heroic age of the struggle
against state censorship of books in the United States was already over.
In 1982, rather than the State of New York seeking to prevent folks from
reading \emph{Ulysses}, we find
\href{https://news.google.com/newspapers?id=9CsdAAAAIBAJ\&sjid=RqUEAAAAIBAJ\&pg=6717\%2C5959414}{the
Moral Majority complaining about works like \emph{Our Bodies, Our
Selves}}. This concern with women's sexuality and health is uncannily
recalled when earlier this month a
\href{http://www.wbir.com/story/news/local/2015/09/07/local-mom-objects--controversial-book--summer-reading-list/71843596/}{Knoxville
parent complained} about the explict references to women's bodies in the
\emph{The Immortal Life of Henrietta Lacks}. Is this debate about
women's sexual health and knowledge, either in the early 1980s or now,
best understood as a debate about books? Or does it have more in common
with a history that, as this moment, materializes as an effort to defund
Planned Parenthood?

Schingler writers, ``Reading only about people our parents and pastors
are comfortable with isn't an education, it's an echo-chamber.'' I
agree. Reading is a wonderful and potentially transformative experience.
It should be celebrated and defended zealously. But if we find people
seeking to limit access to books, we may wonder whether their target is
\emph{books} per se, or something else: \emph{public education} or
\emph{women's health}, both of which require a well-funded state.
Schingler writes, ``Libraries are a marketplace of ideas, and if they're
going to operate in a truly democratic fashion, all ideas should be
represented.'' May be. But the arguments of would-be book-banners are,
right now, often couched exactly in market terms---not that this or that
book should not be published or legally allowed to circulate, but that
\emph{my tax dollars} shouldn't have to pay for it. We love the version
of this conflict which is a struggle between freedom and censorship; but
the conflict today is precisely one which takes place through appeals to
market values---not between freedom and suppression, but \emph{what} to
fund according to what criteria. The real argument today seems less
about ``freedom,'' than about our willingness to fund and maintain a
robust sense of ``public goods.''

As a matter of rhetoric and political tactics, it perhaps makes sense to
throw the weight of a long historical struggle against state censorship
behind our own moment of squabbles in local school boards or funding
lines in state budgets. We should be careful, though, that such rhetoric
doesn't lead us to mistake a fight about public education or women's
health or the rights of queer people for \emph{the right to read}.
Indeed, if we could add a little nuance and history to our sense of the
long struggle to publish controversial books, we might even realize that
the history of books and their banning is already replete with lessons
for these distinct, but not unrelated, struggles (see, for instance, the
case of \emph{The Well of Loneliness} and its banning in England).

So, two cheers for Banned Books Week and for all efforts to protect the
freedom to read. The fullest possible access to the textual record is
indeed a public good worthy of our time, attention, and dedication. It
is not, though, the only good; it may not, at this moment, even be the
most pressing one.

\end{document}
