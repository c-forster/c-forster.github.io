% Options for packages loaded elsewhere
\PassOptionsToPackage{unicode}{hyperref}
\PassOptionsToPackage{hyphens}{url}
%
\documentclass[
  12pt,
]{article}
\usepackage{lmodern}
\usepackage{amssymb,amsmath}
\usepackage{graphicx}
\usepackage{ifxetex,ifluatex}
\ifnum 0\ifxetex 1\fi\ifluatex 1\fi=0 % if pdftex
  \usepackage[T1]{fontenc}
  \usepackage[utf8]{inputenc}
  \usepackage{textcomp} % provide euro and other symbols
\else % if luatex or xetex
  \usepackage{unicode-math}
  \defaultfontfeatures{Scale=MatchLowercase}
  \defaultfontfeatures[\rmfamily]{Ligatures=TeX,Scale=1}
  \setmainfont[]{Hoefler Text}
  \setsansfont[]{Gill Sans}
\fi
% Use upquote if available, for straight quotes in verbatim environments
\IfFileExists{upquote.sty}{\usepackage{upquote}}{}
\IfFileExists{microtype.sty}{% use microtype if available
  \usepackage[]{microtype}
  \UseMicrotypeSet[protrusion]{basicmath} % disable protrusion for tt fonts
}{}
\makeatletter
\@ifundefined{KOMAClassName}{% if non-KOMA class
  \IfFileExists{parskip.sty}{%
    \usepackage{parskip}
  }{% else
    \setlength{\parindent}{0pt}
    \setlength{\parskip}{6pt plus 2pt minus 1pt}}
}{% if KOMA class
  \KOMAoptions{parskip=half}}
\makeatother
\usepackage{xcolor}
\IfFileExists{xurl.sty}{\usepackage{xurl}}{} % add URL line breaks if available
\IfFileExists{bookmark.sty}{\usepackage{bookmark}}{\usepackage{hyperref}}
\hypersetup{
  pdftitle={English Department Undergraduate Courses},
  hidelinks,
  pdfcreator={LaTeX via pandoc}}
\urlstyle{same} % disable monospaced font for URLs
\usepackage[paperheight=8.5in,paperwidth=5.5in,bottom=0.75in,top=0.75in,left=0.5in,right=0.5in]{geometry}
\setlength{\emergencystretch}{3em} % prevent overfull lines
\providecommand{\tightlist}{%
  \setlength{\itemsep}{0pt}\setlength{\parskip}{0pt}}
\setcounter{secnumdepth}{-\maxdimen} % remove section numbering
\ifluatex
  \usepackage{selnolig}  % disable illegal ligatures
\fi


\newlength{\cslhangindent}
\setlength{\cslhangindent}{1.5em}
\newlength{\csllabelwidth}
\setlength{\csllabelwidth}{3em}
\newlength{\cslentryspacingunit} % times entry-spacing
\setlength{\cslentryspacingunit}{\parskip}
\newenvironment{CSLReferences}[2] % #1 hanging-ident, #2 entry spacing
 {% don't indent paragraphs
  \setlength{\parindent}{0pt}
  % turn on hanging indent if param 1 is 1
  \ifodd #1
  \let\oldpar\par
  \def\par{\hangindent=\cslhangindent\oldpar}
  \fi
  % set entry spacing
  \setlength{\parskip}{#2\cslentryspacingunit}
 }%
 {}
\usepackage{calc}
\newcommand{\CSLBlock}[1]{#1\hfill\break}
\newcommand{\CSLLeftMargin}[1]{\parbox[t]{\csllabelwidth}{#1}}
\newcommand{\CSLRightInline}[1]{\parbox[t]{\linewidth - \csllabelwidth}{#1}\break}
\newcommand{\CSLIndent}[1]{\hspace{\cslhangindent}#1}

\usepackage{multicol}


\title{On ``Teaching'' ``Digital Humanities''}
\author{}
\date{}

\begin{document}

As this academic year warms up, some thoughts on the last one; last year
I had the unexpected opportunity to teach a graduate seminar in the
Spring.{I was not as diligent as a blog coordinator, keeping up with
summary posts, as I would have liked, but some summaries and links to
student blogs are available \href{http://630dh.cforster.com/}{here}.} I
wavered between a theories of modernism course (think: Hugh Kenner,
Peter Burger, Frederic Jameson, Susan Stanford Friedman) and an ``Intro
DH'' class, settling on the latter simply because I thought it would be
more valuable to graduate students (few of whom, in our department, have
strong research interests in modernist studies).

The course benefited from a number of sources; one is Scott Weingart's
excellent \href{http://www.scottbot.net/HIAL/?page_id=21794}{list of DH
syllabi}.{He says syllabi; I say syllabuses; it is a battle that has
\href{http://books.google.com/ngrams/graph?content=syllabi\%2Csyllabuses\&year_start=1800\&year_end=2000\&corpus=15\&smoothing=3\&share=}{raged
for centuries}.} I should also thank a number of people who were kind
enough to offer thoughts (and sometimes texts) as I put together the
syllabus: Stéfan Sinclair, David Golumbia, and Brian Lennon offered
suggestions. They were all more than generous; though, of course, they
bear no responsibility for whatsoever for the syllabus. I also was
fortunate enough to end up corresponding with a number of other folks
during the semester, thanks to all of them.

If you're interested in seeing the syllabus, you can see it in
\href{/files/eng630_syllabus-final.pdf}{{[}PDF{]}},
\href{/files/eng630_syllabus-final.html}{{[}HTML{]}}, or (heaven help
you) on \href{https://github.com/c-forster/eng630-syllabus}{GitHub}.{In
theory, the whole github syllabuses thing sounds promising; and for DH
stuff, who knows? But really, just putting the stuff on the web is
probably the best way to share teaching materials.}

The syllabus includes, at least to some extent, basically all the texts
that, a while back, Brian Croxall mentioned as the ``usual'' DH reading
list:

(\textbf{digiwonk?}) (\textbf{readywriting?}) Graphs, Maps, Trees would
be first. Then Debates, two Blackwell books. Ramsay, Jockers,
Kirschenbaum. The usual.

--- Brian Croxall ((\textbf{briancroxall?})) July 19, 2013

I would have \emph{loved} to use Jockers's \emph{Macroanalysis}, but it
was not out in time; I would certainly use it were I to teach the class
again. Indeed, I could very easily imagine a class taught around
\emph{Macroanalysis} as its central text.

I imagined the course as centered by a basically epistemological
perspective: does the ``becoming digital of textuality'' {``becoming
digital of textuality'' is clumsy; but I think it gets at the thing I'm
interested in better than anything else.} change the sorts of knowledge
about literature and culture that scholars produce. Does it offer, to
put it more polemically, a \emph{science} of culture? (I avoided this
polemical formulation in class; not least because of the definition of
``science'' that it assumes). This was the motivation for starting with
C.P. Snow's ``The Two Cultures,'' a text {I would, in the future,
however excerpt Snow; there are a lot more relics of the Cold War in
that essay than I recalled, not all of which were directly relevant.}
which I sometimes see condescendingly referred to as if it were
backward, outdated, or self-evidently wrong, when I find its core thesis
remains provocative and at least partially compelling.

The course was then organized around questions of digitization and
textual representation{The Latour and Lowe essay, from Switching Codes
is a real gem, and one I don't see mentioned very frequently.}, and then
a whole slew of things I called distant reading.

I'm not sure that there's anything in my life that I'd call an
unmitigated success, and this class is surely no exception. I think
though that it did the job of familiarizing folks with at least some of
what ``DH'' is, particularly for folks in English departments.

I'm not sure if I'll ever teach such a course again. But here are some
things I think I learned---things I'd change and things I'd do the same
way again:

\begin{itemize}
\item
  \textbf{More, not less, technical:} I was acutely aware of the worry
  among students that this class would be ``highly technical'' and
  require students to have all sorts of prerequisite knowledge. To avoid
  that, I think I erred too far on the other side, and set the bar too
  low. I integrated some tools into class, while leaving the more
  technical, hands-on stuff for suppelementary (optional) hands-on
  sessions (we did a little Intro Python \& nltk; we did some topic
  modeling with MALLET, some web-scraping, etc). This was logistically a
  problem (finding a time that worked for everyone). More fundamentally,
  my sense is that the class would have been stronger had it been
  \emph{more} technical. The evaluations seem to confirm this almost
  unanimously; everyone thought more hands on with software would be a
  benefit.

  What would that mean in practice? While we spent time talking about
  encoding; and looking at some examples of TEI encoded documents
  (including the \href{http://www.folgerdigitaltexts.org/}{Folger's}),
  we didn't actually encode anything. But until you've had to complete a
  teiHeader, you don't really know the burden of metadata. Actually
  making folks encode a text would be one avenue I might pursue. By
  selecting texts with a sufficiently interesting textual history and
  encoding some apparatus, this could be interesting assignment indeed.

  \href{http://mattwilkens.com/2012/12/31/dh-grad-course-reflections/}{Matt
  Wilkens reports} having success using
  \href{http://programminghistorian.org/lessons}{``The Programming
  Historian''}; I could imagine doing something similar, and centering
  the class's practical activities around Python (about which more
  below); I would do so despite serious reservations about both my own
  qualification to be teaching such material (I am, after all, an
  autodidact in these matters) and the utility that a superficial
  command of a scripting language would give a graduate student in
  English.
\item
  \textbf{One Tool Well}: This is a corollary to the previous point;
  more technical, but also more focused. As we moved through the
  semester, the variety of different software packages we looked at
  increased: basic word frequencies with command line tools, R
  (particularly for mapping), MALLET, Stanford's Named Entity
  Recognizer, Python (with a number of libraries---particularly the
  NLTK), ImageJ, and others. There is a value to examining a diversity
  of software tools; but its costs, upon reflection, now seem too high.
  For someone coming to such technologies for the very first time, I
  think focusing on one, very flexible technology to do all (or at least
  most) of the things we would be interested in doing might be the best
  approach. And Python could fit the bill; certain things may be less
  pleasant in Python, but overall the consistency of a single language
  and syntax would have been a virtue.

  (I will say that while I think I'd prefer Python over R, \emph{if you
  were to use R}, running
  \href{http://www.rstudio.com/ide/docs/server/getting_started}{RStudio
  Server} would be a great way to provide a consistent software base for
  students; as a piece of software, it was pretty stellar. Speaking of
  which\ldots)
\item
  \textbf{Running a Server? Totally Worth It}: One of problems in a
  class like this is infrastructure; you want folks to be able to play
  with some of these tools, but trying to get MALLET, or Python plus a
  handful of libraries, installed on students' machines can be very
  unpleasant. To create some consistency in the software available to
  folks and to avoid having to try to install software packages on 6
  different OS versions and hardware platforms, I rented some server
  space with Linode (approximate cost: \$25/month) and set up user
  accounts for everyone. Then I installed all the software packages we
  would be interested in using.

  Such a setup utterly lacks any GUI (unless you count browser access to
  something like RStudio), which requires folks to be comfortable at the
  command line. This was no small thing. Use of the server was
  essentially optional, and some folks simply never got interested in
  it. In the spirit of ``more, not less, technical'' I'd probably
  require more engagement with the server in a second version. I'll
  write up a more detailed explanation of the server set up I used soon;
  but this worked, over all, very well, and with some tweaks could work
  even better.
\item
  \textbf{I'd Probably Require Twitter}: I suggested folks have a
  twitter account, but didn't require it. The folks who were on twitter,
  though benefited from it (I think); it provided an additional shared
  context; when authors we were reading were on twitter or had blogs, I
  was sure to note it, and I sometimes saw the extra context folks had
  gleaned from these resources in their contributions to class.
\end{itemize}

When I planned the syllabus, I was concerned to try to integrate
criticism of ``DH'' into this class, to make it both a class
\emph{about} ``digital humanities'' as well as a digital humanities
class. I wanted to leaven ambient excitement or ``buzz'' with
skepticism, to use the tools but to do so with care and reflection, to
balance the hacking with yacking (to invoke a short-hand that has
produced much hand-wringing). I will say, however, that during class
meetings, I generally found myself working harder to overcome the
skepticism, rather than to contain the excitement; wanting there to
indeed be a little less yack and a little more\ldots{}

\end{document}
