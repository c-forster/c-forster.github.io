% Options for packages loaded elsewhere
\PassOptionsToPackage{unicode}{hyperref}
\PassOptionsToPackage{hyphens}{url}
%
\documentclass[
  12pt,
]{article}
\usepackage{lmodern}
\usepackage{amssymb,amsmath}
\usepackage{graphicx}
\usepackage{ifxetex,ifluatex}
\ifnum 0\ifxetex 1\fi\ifluatex 1\fi=0 % if pdftex
  \usepackage[T1]{fontenc}
  \usepackage[utf8]{inputenc}
  \usepackage{textcomp} % provide euro and other symbols
\else % if luatex or xetex
  \usepackage{unicode-math}
  \defaultfontfeatures{Scale=MatchLowercase}
  \defaultfontfeatures[\rmfamily]{Ligatures=TeX,Scale=1}
  \setmainfont[]{Hoefler Text}
  \setsansfont[]{Gill Sans}
\fi
% Use upquote if available, for straight quotes in verbatim environments
\IfFileExists{upquote.sty}{\usepackage{upquote}}{}
\IfFileExists{microtype.sty}{% use microtype if available
  \usepackage[]{microtype}
  \UseMicrotypeSet[protrusion]{basicmath} % disable protrusion for tt fonts
}{}
\makeatletter
\@ifundefined{KOMAClassName}{% if non-KOMA class
  \IfFileExists{parskip.sty}{%
    \usepackage{parskip}
  }{% else
    \setlength{\parindent}{0pt}
    \setlength{\parskip}{6pt plus 2pt minus 1pt}}
}{% if KOMA class
  \KOMAoptions{parskip=half}}
\makeatother
\usepackage{xcolor}
\IfFileExists{xurl.sty}{\usepackage{xurl}}{} % add URL line breaks if available
\IfFileExists{bookmark.sty}{\usepackage{bookmark}}{\usepackage{hyperref}}
\hypersetup{
  pdftitle={English Department Undergraduate Courses},
  hidelinks,
  pdfcreator={LaTeX via pandoc}}
\urlstyle{same} % disable monospaced font for URLs
\usepackage[paperheight=8.5in,paperwidth=5.5in,bottom=0.75in,top=0.75in,left=0.5in,right=0.5in]{geometry}
\setlength{\emergencystretch}{3em} % prevent overfull lines
\providecommand{\tightlist}{%
  \setlength{\itemsep}{0pt}\setlength{\parskip}{0pt}}
\setcounter{secnumdepth}{-\maxdimen} % remove section numbering
\ifluatex
  \usepackage{selnolig}  % disable illegal ligatures
\fi


\newlength{\cslhangindent}
\setlength{\cslhangindent}{1.5em}
\newlength{\csllabelwidth}
\setlength{\csllabelwidth}{3em}
\newlength{\cslentryspacingunit} % times entry-spacing
\setlength{\cslentryspacingunit}{\parskip}
\newenvironment{CSLReferences}[2] % #1 hanging-ident, #2 entry spacing
 {% don't indent paragraphs
  \setlength{\parindent}{0pt}
  % turn on hanging indent if param 1 is 1
  \ifodd #1
  \let\oldpar\par
  \def\par{\hangindent=\cslhangindent\oldpar}
  \fi
  % set entry spacing
  \setlength{\parskip}{#2\cslentryspacingunit}
 }%
 {}
\usepackage{calc}
\newcommand{\CSLBlock}[1]{#1\hfill\break}
\newcommand{\CSLLeftMargin}[1]{\parbox[t]{\csllabelwidth}{#1}}
\newcommand{\CSLRightInline}[1]{\parbox[t]{\linewidth - \csllabelwidth}{#1}\break}
\newcommand{\CSLIndent}[1]{\hspace{\cslhangindent}#1}

\usepackage{multicol}


\title{From New Hampshire to Harlem, by Way of London}
\author{}
\date{}

\begin{document}

While I haven't been vocal about it, work has continued, in off moments
and stolen time, on the online edition of Claude McKay's \emph{Harlem
Shadows} which I described \href{/2012/06/drill-baby-drill/}{some time
ago}. At the present moment, \href{http://roopikarisam.com/}{Roopika
Risam} and I have collected nearly all the textual variants and have
marked them up in TEI; we added (as yet unproofread) versions of early
reviews and other supplemental material (and still more is being hunted
down and added); and there is enough XSLT and CSS to hold the whole
thing together, more or less. It is very much still a work in progress,
but you can see the current state of its progress
\href{http://harlemshadows.org/beta/}{here}.

This process has also been an opportunity to understand the textual
history of the poems of \emph{Harlem Shadows}, including the
relationship of the collection \emph{Harlem Shadows} to McKay's earlier
collection \emph{Spring in New Hampshire}. The Jamaican poet who travels
to rural Kansas in order to pursue a degree in agriculture and ends up
being one of the early voices of Harlem Renaissance, manages to do so by
passing through not only Harlem, but New Hampshire and, crucially,
London. \emph{Spring in New Hampshire} was how many readers first
encountered McKay (including readers like Charlie Chaplin and Hubert
Harrison), and the collection offers a valuable first draft of
\emph{Harlem Shadows}.

The collection \emph{Spring in New Hampshire} was first published in
1920. Its ``Acknowledgments'' page notes two facts which underscore this
volume's importance in the emergence of \emph{Harlem Shadows}.

First, when \emph{Spring in New Hampshire} appeared, an American edition
of was clearly imagined as immiment. But the American edition,
purportedly ``being published simultaneously by Alfred A. Knopf,'' never
materialized. What did appear, two years later (published by Harcourt,
Brace, and Co) was \emph{Harlem Shadows}.

And if \emph{Harlem Shadows} is substantially indebted to \emph{Spring
in New Hampshire} {About one third of \emph{Harlem Shadows}'s poems
appear in \emph{Spring}, among them ``Tropics in New York,'' ``The
Barrier,'' ``North and South,'' ``Harlem Shadows,'' ``The Harlem
Dancer,'' and ``The Lynching''.}, \emph{Spring in New Hampshire} in turn
is less an origin than another gathering point for poems culled from
elsewhere; this is especially the case of a large selection of poems
which appear in the Summer 1920 issue of \emph{The Cambridge Magazine}.
This latter includes 23 of \emph{Spring in New Hampshire}'s 31 poems.
And, with the exception of the dedication of ``Spring in New Hampshire''
(dedicated in \emph{Spring} to ``J. L. J. F. E.''{This would almost
certainly be Dutch bibliophile, and the man in part responsible for
McKay's trip to London, J. L. J. F. Ezerman (Gosciak 117).}), there are
\emph{no textual differences} between the poems as they appear in
\emph{CM} and as they appear in \emph{Spring}.

To secure the point I'm moving towards, compare these images, taken from
the appearance of ``The Tropics in New York'' in \emph{Cambridge
Magazine} (top) and \emph{Spring in New Hampshire} (bottom): {My thanks
to \href{http://twitter.com/nickmimic}{Nicholas Morris} who takes no
responsibility for this conjecture, but was enormously helpful in
discussing its plausibility.}

Do you see that imperfection in the `I' of ``I could no more gaze'' in
both versions. Does they look identical to you too? It seems reasonable
to conjecture that the \emph{Cambridge Magazine} poems and \emph{Spring
in New Hampshire} were both printed, if not from a single setting of
type, then at least from a setting of type which likely included some of
the same typeset material (in either monotype, linotype, or set by hand)
from the \emph{Cambridge Magazine}.{In the interest of full disclosure,
there is a
\href{/images/harlem-shadows/when-dawn_comparison.png}{similar
imperfection} in the \emph{Cambridge Magazine} text of ``When Dawn Comes
to the City'' which does not appear in the \emph{Spring in New
Hampshire}; but this does not vitiate the possibility, and evidence,
suggesting the two texts represent something like a single setting of
type.}

The circumstances surrounding \emph{Cambridge Magazine} likewise seem to
confirm this possibility. \emph{Cambridge Magazine}, at the time, was
run by C. K. Ogden, with whom McKay spent time while visiting England in
1920. Ogden ran the magazine in collaboration with a number of his
friends (among them, I. A. Richards). Of Ogden, McKay would write in his
autobiography \emph{A Long Way from Home}: ``besides steering me round
the picture galleries and being otherwise kind, {[}Ogden{]} had
published a set of my verses in his \emph{Cambridge Magazine}. Later he
got me a publisher'' (71). If McKay means that Ogden secured the
publisher for \emph{Spring in New Hampshire} (and that seems the most
likely meaning here), it would certainly make sense that Ogden would go
through the same publishing channels (including, perhaps, the same
printer) as for the periodical for which he was responsible. {The
frontmatter of \emph{Spring in New Hampshire} (published by Grant
Richards), lists the printer as ``The Morland Press.''} And while Ogden
(according to Josh Gosciak) authored the prefatory note for the
appearance of the poems in \emph{Cambridge Magazine}, it was I. A.
Richards (a friend of Ogden's, who regularly appeared in the
\emph{Cambridge Magazine}, including the Summer 1920 issue in which
McKay's poems appeared) who wrote the note the \emph{Spring in New
Hampshire} (after, according to McKay, George Bernard Shaw declined to
write such an introductory note, \emph{Long Way Home} 55).

All of which is interesting and worthy of note insomuch as it suggests
that \emph{Harlem Shadows}, key document of the Harlem Renaissance, has
its origins not only in Jamaica and New York, but in New Hampshire and
London. This eclecticism was vital to Ogden's interest in McKay; Ogden
was at this moment, working with I. A. Richards on what would emerge as
``Basic English.'' In
\href{http://en.wikipedia.org/wiki/Basic_English}{BASIC}, ``Ogden wanted
a usable language that reflected the hybridity of the changing dynamic
of cultures and languages in the Caribbean, Africa, the United States,
and Asia'' (Gosciak 102). And in McKay, Ogden believed he had found a
uniquely valuable voice in the development of such a language---a
language, Gosciak describes, which would ``decolonize the dominant
ideology that espoused war and imperialism'' (103).

Yet, the way in which McKay and Ogden imagined this decolonization of
English is somewhat surprising. McKay came to Ogden frustrated with what
he perceived to be the limitations of his previously published poetry,
in Jamacian dialect. Here is Gosciak again:

\begin{quote}
{[}McKay's{]} reputation was as the ``Bobby Burns'' of Jamaican folk
wisdom, who could write persuasive ``love songs'' in a sonorous dialect.
But an exasperated McKay explained to Ogden: ``One can't express any
deep thought to perfection in it, nor can it effectively bring forth the
note of sorrow.'' Dialect was hackneyed, McKay concluded. ``I've buried
it and don't care to revive it again.'' Ogden was sympathetic to the
poet's desires to internationalize his poetics, and he mentioned him in
precision and exactness---de-emotionalizing his lyrics of the charged
baggage of Harlem and race. (104)
\end{quote}

This tension is manifest in a disagreement between Ogden and McKay over
what to title the collection. Ogden was interested in developing an
international English, Gosciak, drawing on material in the
\href{http://library.mcmaster.ca/archives/findaids/findaids/o/ogden.htm}{Papers
of CK Ogden} explains:

\begin{quote}
McKay was opposed to {[}the title{]} \emph{Spring in New Hampshire, and
Other Poems}; he believed the title conjured associations with New
England, which he felt was ``played out''\ldots{} McKay preferred ``a
terse, simple thing'' for a title, such as ``Poems or Verse,'' which
Ogden, too, appreciated. But McKay also had an eye for the New
York---and Harlem---reading public. He suggested ``Dawn in New York,''
invoking imagery that would ultimately give texture to \emph{Harlem
Shadows} in 1922. Ogden persisted in his claims for the high lyricism of
Frost, and eventually McKay came around to that aesthetic ground. (The
choice of title, \emph{Spring in New Hampshire}, was, as McKay
acknowledged, a bold move for a poet who would very soon reprsent the
Harlem Renaissance.) (Gosciak 105)
\end{quote}

There is sort of confusion of motivations here; McKay's frustration with
dialect and Ogden's attempt to decolonize English both find expression
of in the poems of \emph{Spring in New Hampshire}---poems that rely on
traditional forms---sonnets aplenty!---and frequently Victorian diction.
Yet, Ogden's vision of de-colonizing English also involves
de-racinating, with the effect that Ogden preferred to see McKay's verse
avoid any allusion too direct to Harlem or race.

And so \emph{Spring in New Hampshire} ends up being as notable for what
it \emph{doesn't} share with \emph{Harlem Shadows} as what it does. The
most famous poem of \emph{Harlem Shadows}, ``If We Must Die,'' had first
appeared in \emph{The Liberator} in 1919, but it was not included in
\emph{Spring in New Hampshire}. In \emph{A Long Way Home}, McKay
recounts bringing a copy of \emph{Spring in New Hampshire} to Frank
Harris, of \emph{Pearson's Magazine} (who had wanted to publish ``If We
Must Die,'' though he lost out to \emph{The Liberator}):

\begin{quote}
{[}Harris{]} was pleased that I had put over the publication of a book
of poems in London. ``It's a hard, mean city for any kind of genius,''
he said, ``and that's an achivement for you.'' He looked through the
little brown-covered book. Then he ran his finger down the table of
contents, closely scrutinizing. I noticed his aggressive brow becoem
heavier and scowling. Suddenly he roared: ``Where is the poem?\ldots{}
That fighting poem, `If We Must Die.' Why isn't it printed here?''

I was ashamed. My face was scorched with fire. I stammered: ``I was
advised to keep it out.''

``You are a bloody traitor to your race, sir!'' Frank Harris shouted.
``A damned traitor to your own integrity. That's what the English and
civilization have done to your people. Emasculated them. Deprived them
of their guts. Better you were a head-hunting, blood-drinking cannibal
of the jungle than a civilized coward. You were bolder in America. The
English make obscene sycophants of their subject peoples. I am Irish and
I know. But we Irish have guts you cannot rip out of us. I am ashamed of
you, sir. It's a good thing you got out of Engliand. It is no place for
a genius to live.''

Frank Harris's words cut like a whip into my hide, and I was glad to get
out of his uncomfortable presence. Yet I felt relieved after his
castigation. The excision of the poem had been like a nerve cut out of
me, leaving a wound which would not heal. And it hurt more every time I
saw the damned book of verse. I resolved to plug hard for the
publication of an American edition, which would include the omitted
poem. (81-82)
\end{quote}

McKay here ends up being caught between two white editors, and their
respective ways of imagining a response to British colonialism. (This
situation recalls that of McKay and his relationship to dialect poetry
discussed in Michael North's excellent chapter in \emph{The Dialect of
Modernism}.)

That American edition that McKay resolves to publish after this
encounter with Harris would, of course, be \emph{Harlem Shadows}.
\emph{Harlem Shadows}, in McKay's depiction, is a \emph{version} of
\emph{Spring in New Hampshire} and a repudiation of it. Elsewhere in his
autobiography he writes, ``I was full and overflowing with singing and I
sang all moods, wild, sweet, and bitter. I was steadfastly pursuing one
object: the publication of an American book of verse. I desired to see
`If We Must Die,' the sonnet I had omitted in the London volume, inside
of a book'' (116).

Yet, if McKay's comments encourage us to read \emph{Harlem Shadows} as a
re-politicized version of \emph{Spring in New Hampshire}, ``If We Must
Die'' itself, nevertheless, famously operates by abstracting the
political violence of the ``Red Summer'' of 1919 into an unspecified
``we kinsmen'' against a ``common foe.'' And, indeed, \emph{Harlem
Shadows}, like \emph{Spring in New Hampshire}, does not include some of
McKay's most explicitly political poetry of this period---poems like
\href{http://harlemshadows.org/beta/index.html\#supp_mckay_to-the-white-fiends}{``To
the White Fiends''} or ``A Capitalist at Dinner,'' which were initially
published in the same period, and in the same venues, as poems like ``If
We Must Die,'' remain excluded.

All of which indicates the value of a comprehensive collection of all
the contemporary poems and material which went into the making of
\emph{Harlem Shadows}---both through their inclusion and their
exclusion.

\hypertarget{appendix-tables-of-contents-compared}{%
\subsection{Appendix: Tables of Contents
Compared}\label{appendix-tables-of-contents-compared}}

Below I've preserved the original orderings of the tables of contents
for both \emph{Harlem Shadows} and \emph{Spring in New Hampshire} and
used color (a lovely salmon) to indicate which titles are shared.

Harlem Shadows

Spring in New Hampshire

The Easter Flower

Spring in New Hampshire

To One Coming North

The Spanish Needle

America

The Lynching

Alfonso, Dressing to Wait at Table

To O. E. A.

The Tropics in New York

Alfonso, Dressing to Wait at Table, Sings

Flame Heart

Flowers of Passion

Home Thoughts

To Work

On Broadway

Morning Joy

The Barrier

Reminiscences

Adolescence

On Broadway

Homing Swallows

Love Song

The City's Love

North and South

North and South

Rest in Peace

Wild May

A Memory of June

The Plateau

To Winter

After the Winter

Winter in the Country

The Wild Goat

After the Winter

Harlem Shadows

The Tropics in New York

The White City

I Shall Return

The Spanish Needle

The Castaways

My Mother

December 1919

In Bondage

Flame-Heart

December, 1919

In Bondage

Heritage

Harlem Shadows

When I Have Passed Away

The Harlem Dancer

Enslaved

A Prayer

I Shall Return

The Barrier

Morning Joy

When Dawn Comes to the City

Africa

The Choice

On a Primitive Canoe

Sukee River

Winter in the Country

Exhortation

To Winter

Spring in New Hampshire

On the Road

The Harlem Dancer

Dawn in New York

The Tired Worker

Outcast

I Know My Soul

Birds of Prey

The Castaways

Exhortation: Summer, 1919

The Lynching

Baptism

If We Must Die

Subway Wind

The Night Fire

Poetry

To a Poet

A Prayer

When Dawn Comes to the City

O Word I Love to Sing

Absence

Summer Morn in New Hampshire

Rest in Peace

A Red Flower

Courage

To O. E. A.

Romance

Flower of Love

The Snow Fairy

La Paloma in London

A Memory of June

Flirtation

Tormented

Polarity

One Year After

French Leave

Jasmines

Commemoration

Memorial

Thirst

Futility

Through Agony

\hypertarget{works-cited}{%
\section{Works Cited}\label{works-cited}}

\begin{itemize}
\item
  Gosciak, Josh. \emph{The Shadowed Country: Claude Mckay and the
  Romance of the Victorians.} New Brunswick, N.J.: Rutgers University
  Press, 2006. Print.
\item
  McKay, Claude. \emph{A Long Way from Home}. Ed. Gene Andrew Jarrett.
  New Brunswick, N.J.: Rutgers University Press, 2007. Print.
\end{itemize}

\end{document}
