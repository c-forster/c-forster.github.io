% Options for packages loaded elsewhere
\PassOptionsToPackage{unicode}{hyperref}
\PassOptionsToPackage{hyphens}{url}
%
\documentclass[
  12pt,
]{article}
\usepackage{lmodern}
\usepackage{amssymb,amsmath}
\usepackage{graphicx}
\usepackage{ifxetex,ifluatex}
\ifnum 0\ifxetex 1\fi\ifluatex 1\fi=0 % if pdftex
  \usepackage[T1]{fontenc}
  \usepackage[utf8]{inputenc}
  \usepackage{textcomp} % provide euro and other symbols
\else % if luatex or xetex
  \usepackage{unicode-math}
  \defaultfontfeatures{Scale=MatchLowercase}
  \defaultfontfeatures[\rmfamily]{Ligatures=TeX,Scale=1}
  \setmainfont[]{Hoefler Text}
  \setsansfont[]{Gill Sans}
\fi
% Use upquote if available, for straight quotes in verbatim environments
\IfFileExists{upquote.sty}{\usepackage{upquote}}{}
\IfFileExists{microtype.sty}{% use microtype if available
  \usepackage[]{microtype}
  \UseMicrotypeSet[protrusion]{basicmath} % disable protrusion for tt fonts
}{}
\makeatletter
\@ifundefined{KOMAClassName}{% if non-KOMA class
  \IfFileExists{parskip.sty}{%
    \usepackage{parskip}
  }{% else
    \setlength{\parindent}{0pt}
    \setlength{\parskip}{6pt plus 2pt minus 1pt}}
}{% if KOMA class
  \KOMAoptions{parskip=half}}
\makeatother
\usepackage{xcolor}
\IfFileExists{xurl.sty}{\usepackage{xurl}}{} % add URL line breaks if available
\IfFileExists{bookmark.sty}{\usepackage{bookmark}}{\usepackage{hyperref}}
\hypersetup{
  pdftitle={English Department Undergraduate Courses},
  hidelinks,
  pdfcreator={LaTeX via pandoc}}
\urlstyle{same} % disable monospaced font for URLs
\usepackage[paperheight=8.5in,paperwidth=5.5in,bottom=0.75in,top=0.75in,left=0.5in,right=0.5in]{geometry}
\setlength{\emergencystretch}{3em} % prevent overfull lines
\providecommand{\tightlist}{%
  \setlength{\itemsep}{0pt}\setlength{\parskip}{0pt}}
\setcounter{secnumdepth}{-\maxdimen} % remove section numbering
\ifluatex
  \usepackage{selnolig}  % disable illegal ligatures
\fi


\newlength{\cslhangindent}
\setlength{\cslhangindent}{1.5em}
\newlength{\csllabelwidth}
\setlength{\csllabelwidth}{3em}
\newlength{\cslentryspacingunit} % times entry-spacing
\setlength{\cslentryspacingunit}{\parskip}
\newenvironment{CSLReferences}[2] % #1 hanging-ident, #2 entry spacing
 {% don't indent paragraphs
  \setlength{\parindent}{0pt}
  % turn on hanging indent if param 1 is 1
  \ifodd #1
  \let\oldpar\par
  \def\par{\hangindent=\cslhangindent\oldpar}
  \fi
  % set entry spacing
  \setlength{\parskip}{#2\cslentryspacingunit}
 }%
 {}
\usepackage{calc}
\newcommand{\CSLBlock}[1]{#1\hfill\break}
\newcommand{\CSLLeftMargin}[1]{\parbox[t]{\csllabelwidth}{#1}}
\newcommand{\CSLRightInline}[1]{\parbox[t]{\linewidth - \csllabelwidth}{#1}\break}
\newcommand{\CSLIndent}[1]{\hspace{\cslhangindent}#1}

\usepackage{multicol}


\title{The Podcast as a Genre}
\author{}
\date{}

\begin{document}

What precisely is a podcast? I once heard a minimal definition of a
podcast as an mp3 file attached to an RSS feed---which is to say,
syndicated audio content on the internet. But looking around, there are
plenty of podcasts that don't meet this criteria: podcasts that lack an
RSS feed (WHY?!?), to speak nothing of ``video podcasts'' (which people
are apparently strill trying to make happen). ``Podcast'' can sometimes
be used as a verb to mean something like ``transmitting audio over the
internet'' (e.g.~``Will you be podcasting that keynote lecture?'').
Looking at iTunes, you realize plenty of ``podcasts'' are just radio
shows put on the internet: iTunes's most popular podcasts are mostly
public radio fare (like ``This American Life'' and ``Radiolab'').

But, the podcast is not simply a technology or a channel. I've been
listening to podcasts for awhile now and have been curious to watch my
habits slowly shift, moving away from ``radio shows on the internet''
(\emph{Fresh Air}, whenever I want it!) to something else.
\href{http://niemanstoryboard.org/stories/finding-the-tribe/}{This
piece} looks at the ``return'' of podcasts as a medium, mostly
considering the podcast as a business model. It does however offer this,
from ``Planet Money'' podcaster Alex Blumberg, on what makes podcasts
different:

\begin{quote}
``It's the most intimate of mediums. It's even more intimate than radio.
Often you're consuming it through headphones. I feel like there's a bond
that's created.''
\href{http://niemanstoryboard.org/stories/finding-the-tribe/}{Source}
\end{quote}

That seems entirely right to me, and it helpfully points to some of the
ways that what I'll call podcasts \emph{as a genre} differ from
understanding podcasts as just ``radio over the internet.'' The
``podcast'' as a form blurs the line between a medium (say, a recurring,
asynchronously consumed type of audio---usually neither music or
fiction) and a genre. The podcast, as medium, has been enabled by
readier access to bandwidth, software technologies like iTunes
syndication and RSS, and developments in hardware like relatively cheap
but entirely decent microphones{Woe unto the podcaster who relies on
built-in mics on laptops and phones, for he shall receive low traffic.}
and of course the iPod. But these technologies, in their use, create a
sort of gravitational pull toward a form that is less formal, more
niche, and therefore oddly closer to a sort of specialized and
heightened mode of casual conversation than it is to most radio genres.

When the costs of creating and distributing recordings of folks talking
into microphones gets \emph{way} cheaper than the costs of
writing/producing/reporting stories, you get a new sort of show---where
folks just sit around and talk. Central to the conventions of this genre
is, I think, the group of regular or semi-regular folks who sit around
and talk about something. Such are Leo Laporte's
\href{http://www.twit.tv}{TWIT} podcasts; the original TWiT, one of the
first podcasts I listened to, was indeed Leo Laporte sitting with folks
(some of whom his listeners recognize as, like Laporte, erstwhile TechTV
employees) and talking about the week's technology news. This form tends
to be parasitic on some other type of content---on news or culture
(\href{http://www.tommerritt.com/category/shows/daily-tech-news-show/}{daily}
or
\href{http://www.slate.com/articles/podcasts/culturegabfest.html}{weekly}
or \href{http://digitalcampus.tv}{semi-regularly}), or even on a
specific film or primary text. There has to be some \emph{reason}, some
excuse or alibi, for the conversation to exist---but the podcast offers
a conversation rather than the news.

This may not seem especially novel---after all, personality-driven
``analysis'' now dominates cable news. Yet cable news analysis shows
usually center on a single individual, and their dominant moods are
outrage or indignation or derision; they tend to be centered \textbf{a
personality} (variably likeable or not) who offers a ``perspective.''
But what a podcast offers is not a perspective (or not \emph{chiefly} a
perspective) but something more like a performance of community. In
place of the singular personality, we get personalities. A podcast tends
to create characters, or caricatures, out of its hosts: for instance,
Stephen Metcalf's snobbish nostalgia for the world of print clashing
regularly with Julia Turner's culturally omnivorous techno-utopianism on
the
\href{http://www.slate.com/articles/podcasts/culturegabfest.html}{Slate
Culturefest} (both, of course, unfair exagerations). But in other
podcasts (perhaps notably, podcasts not affiliated with any large online
media presence), this develops into a sense of shared
reference---something like \emph{insiderness} or \emph{knowingness}. The
result is that certain podcasts (the podcastiest of the podcasts by my
sense of the genre) rely heavily on inside jokes. Consider the following
short phrases: ``Who the hell is Casey?''; ``Does this look clean to
you?''; ``The Port Hole of Time.'' To the listeners of certain podcasts,
they will immediately register as inside jokes---from, respectively:
\href{http://atp.fm}{The Accidental Tech Podcast};
\href{http://5by5.tv/b2w}{Back to Work} (quoting the film \emph{The
Aviator}, which in the universe of \emph{Back to Work} is frequently
referered to as simply \emph{the film}); and
\href{http://www.flophousepodcast.com/}{The Flop House}. Listeners of
these podcasts (and I listen to all of these pretty faithfully, though
the truly faithful will likely fault my selections) come to recognize
these, and participate in the joke. These podcasts create a universe of
reference alienating to the newcomer, but comforting to the regular. And
the result is just wonderful. These are my guiltiest of guilty pleasure.
I try to conceal my love for them, but I cannot.

That intimacy of the medium described by Alex Blumberg, created by the
circumstances of consumption (on headphones or in the car{Are
\href{http://www.amazon.com/MP3-Player-Cassette-Adapter-Equipment/dp/B003Q9LRPO}{these
things} great, or what?}), manifests in the genre as a tendency towards
dense self-reference.

The result is that the topic of the podcast can increasingly seem to be
just an alibi for the interactions of its hosts. I don't really care
about Apple News, but listen to \href{http://atp.fm}{ATP} regularly. The
greatest joy of \emph{The Flop House} (a ``bad movie'' podcast, which
reviews/discussions relatively recent theatrical ``flops'') is the
experience of hearing the hosts \emph{summarize the plot of a movie} and
the digressions that ensue. One emphatically does not have to have seen
the movie to enjoy the podcast, and unlike a review (or even the
discussions of film and TV on the \emph{Slate Culturefest}), it is
completely beside the point whether you will see the movie at some point
in the future. I suspect that I'll never see the
\href{http://www.imdb.com/title/tt0804452/}{\emph{Bratz} movie}; but I
shall cherish all the days of my life
\href{http://www.flophousepodcast.com/2008/04/episode-14-bratz/}{\emph{The
Flop House}'s discussion of it}. Listen to early episodes and you'll see
that the plot summary initially presented a challenge---something they
glossed over or tried to get past in order to get to the discussion (on
at least one occassion they just read the Wikipedia summary of a movie).
But the joy of the show is entirely in the interactions between its
hosts, and so something as rote as a plot summary becomes the perfect
opportunity for such interaction. It also explains why at least I find
these sorts of shows more engaging than other audio content. The
academic lecture, or even \emph{Fresh Air}-style interviews, sometimes
allows distraction. But the developing conversation, and tissue of
self-reference, simulates the experience of interaction rather than,
say, the communication of information. (What an interview show like
\emph{Fresh Air} lacks is the regularity of its participants; you're
usually learning something \emph{about} a guest rather than a
conversation between people who already know each other.)

By foregrounding in jokes and habits of communication, the podcast turns
out to be a cousin to that other ``internetiest'' of forms: the meme.
The meme is likewise an in-joke, where the in-group is those folks who
recognize the meme and understand its conventions. The humor of any
individual \href{http://knowyourmeme.com/memes/doge}{``doge,''} meme
(remember that?) is siphoned off from the larger system of doge memes
that makes any particular meme legible and funny. (A picture of a cat
with some funny, misspelled words, encoutered in utter isolation, carved
into the face of some alien moon millenia hence, would be funny because
absurd---but it wouldn't be a meme and wouldn't participate in its
humor.)

The affective range of the podcast is much wider than that of meme,
chiefly because hearing a conversation between the same set of people
(semi)regularly opens more possibilities than silly pictures and block
letters. (There I said it; call me elitist.) But this affective depth
cuts the other way---it also suggests what I find mildly unsettling
about the form, and perhaps slightly embarassing about my enjoyment of
it. If I'm right that inside jokes, and a certain performance of knowing
insiderness, are what separates the podcast as a genre from its radio
peers, it also feels a little like media consumption as simulated
friendship. Its enjoyments are those of easy familiarity and comfortable
in-jokes, but with friends who aren't yours. (You might call this the
anxiety of authenticity, and I'll just take my lumps for worrying over
something as old-fashioned as authenticity.)

More troublingly, that same affective register (of chummy friendship and
inside jokes) seems downright insidious when you realize how
overwhelmingly the list of podcasts I've cited here is dominanted by
white guys. In so much as the pleasures and affects of the genre are
those associated with those of the proverbial boys club, it is dismaying
to see how much of a boy's club it often is.

What is a podcast? It is the humanization of the internet meme, a type
of low-participation friendship, a reduced agency form of ``hanging
out.''

Yours in Flopitude, Chris {[}Last Name Witheld{]}

\end{document}
