% Options for packages loaded elsewhere
\PassOptionsToPackage{unicode}{hyperref}
\PassOptionsToPackage{hyphens}{url}
%
\documentclass[
  12pt,
]{article}
\usepackage{lmodern}
\usepackage{amssymb,amsmath}
\usepackage{graphicx}
\usepackage{ifxetex,ifluatex}
\ifnum 0\ifxetex 1\fi\ifluatex 1\fi=0 % if pdftex
  \usepackage[T1]{fontenc}
  \usepackage[utf8]{inputenc}
  \usepackage{textcomp} % provide euro and other symbols
\else % if luatex or xetex
  \usepackage{unicode-math}
  \defaultfontfeatures{Scale=MatchLowercase}
  \defaultfontfeatures[\rmfamily]{Ligatures=TeX,Scale=1}
  \setmainfont[]{Hoefler Text}
  \setsansfont[]{Gill Sans}
\fi
% Use upquote if available, for straight quotes in verbatim environments
\IfFileExists{upquote.sty}{\usepackage{upquote}}{}
\IfFileExists{microtype.sty}{% use microtype if available
  \usepackage[]{microtype}
  \UseMicrotypeSet[protrusion]{basicmath} % disable protrusion for tt fonts
}{}
\makeatletter
\@ifundefined{KOMAClassName}{% if non-KOMA class
  \IfFileExists{parskip.sty}{%
    \usepackage{parskip}
  }{% else
    \setlength{\parindent}{0pt}
    \setlength{\parskip}{6pt plus 2pt minus 1pt}}
}{% if KOMA class
  \KOMAoptions{parskip=half}}
\makeatother
\usepackage{xcolor}
\IfFileExists{xurl.sty}{\usepackage{xurl}}{} % add URL line breaks if available
\IfFileExists{bookmark.sty}{\usepackage{bookmark}}{\usepackage{hyperref}}
\hypersetup{
  pdftitle={English Department Undergraduate Courses},
  hidelinks,
  pdfcreator={LaTeX via pandoc}}
\urlstyle{same} % disable monospaced font for URLs
\usepackage[paperheight=8.5in,paperwidth=5.5in,bottom=0.75in,top=0.75in,left=0.5in,right=0.5in]{geometry}
\setlength{\emergencystretch}{3em} % prevent overfull lines
\providecommand{\tightlist}{%
  \setlength{\itemsep}{0pt}\setlength{\parskip}{0pt}}
\setcounter{secnumdepth}{-\maxdimen} % remove section numbering
\ifluatex
  \usepackage{selnolig}  % disable illegal ligatures
\fi


\newlength{\cslhangindent}
\setlength{\cslhangindent}{1.5em}
\newlength{\csllabelwidth}
\setlength{\csllabelwidth}{3em}
\newlength{\cslentryspacingunit} % times entry-spacing
\setlength{\cslentryspacingunit}{\parskip}
\newenvironment{CSLReferences}[2] % #1 hanging-ident, #2 entry spacing
 {% don't indent paragraphs
  \setlength{\parindent}{0pt}
  % turn on hanging indent if param 1 is 1
  \ifodd #1
  \let\oldpar\par
  \def\par{\hangindent=\cslhangindent\oldpar}
  \fi
  % set entry spacing
  \setlength{\parskip}{#2\cslentryspacingunit}
 }%
 {}
\usepackage{calc}
\newcommand{\CSLBlock}[1]{#1\hfill\break}
\newcommand{\CSLLeftMargin}[1]{\parbox[t]{\csllabelwidth}{#1}}
\newcommand{\CSLRightInline}[1]{\parbox[t]{\linewidth - \csllabelwidth}{#1}\break}
\newcommand{\CSLIndent}[1]{\hspace{\cslhangindent}#1}

\usepackage{multicol}


\title{Public Domain Editions}
\author{}
\date{}

\begin{document}

This is an extended version of the (less than) two minute ``dork short''
or ``lightning talk'' I gave at THATCamp Virginia a while ago (this post
has been sitting in the hopper for a while). I offer an observation, an
anecdote, and a suggestion.

\textbf{tl;dr}: I'm trying to put together an edition of Claude McKay's
\emph{Harlem Shadows.} Would you like to help?

\hypertarget{an-observation}{%
\subsection{An Observation:}\label{an-observation}}

An enormous wealth of public domain material is available on the web,
from sources like \href{http://gutenberg.org}{Project Gutenberg} or
\href{http://ota.ahds.ac.uk/}{The Oxford Text Archive} or
\href{http://archive.org}{The Internet Archive} or
\href{http://books.google.com}{Google Books} or smaller projects like
\href{http://dl.lib.brown.edu/mjp/}{The Modernist Journals Project}.

Yet, in my experience, these texts seem underused. (Am I wrong?)

\hypertarget{an-anecdote}{%
\subsection{An Anecdote:}\label{an-anecdote}}

When I was a teaching assistant for UVA's twentieth-century literature
survey a few years ago, the professors taught Claude McKay's
\emph{Harlem Shadows}. Published in 1922, \emph{Harlem Shadows} is just
inside the public domain.

The text they used was a cheap (though still in the neighborhood of
\$15; here it is at
\href{http://www.amazon.com/Harlem-Shadows-Poems-Claude-Mckay/dp/1120198720/}{Amazon})
paperback facsimile of the 1922 edition. When I opened this slight
paperback, it looked eerily familiar.

Compare:

The top image is from the Google Books edition; the bottom is a scan I
just made of the Kessinger edition. Kessinger's ``edition'' of
\emph{Harlem Shadows} is printed from page images available at
\href{http://books.google.com/books?id=aKTPAAAAMAAJ}{Google Books}
(scanned, in turn, from a copy at Indiana University library). They've
cleaned up the title page a bit, but look at the distinct pencil marks.
That's Kessinger's business: get new ISBNs for Google Books scans and
then sell them. (When folks first noticed Kessinger doing this a while
ago it caused
\href{http://productforums.google.com/forum/\#!category-topic/books/google-books-searching-and-discovering/4b8Bj0HTtxE}{some
consternation}.)

(Worth noting: there is a
\href{http://books.google.com/books?id=EVFBAAAAYAAJ}{another copy} of
\emph{Harlem Shadows} (scanned from a copy held at Princeton) in GBooks,
which misidentifies Max Eastman in the author metadata; in addition to
the two Google Books copies, archive.org has two copies;
\href{http://www.archive.org/details/harlemshadows00mcka}{one} from the
Library of Congress and
\href{http://www.archive.org/details/harlemshadowspoe00mckauoft}{one}
from the University of Toronto, all the same edition. Thoughts on easily
breaking up those four PDFs and digitally collating them?)

It seems unfortunate that right now a professor who wants to teach
\emph{Harlem Shadows}, ends up assigning Kessinger's rather ugly
print-out of a Google Books PDF.

\hypertarget{a-suggestion}{%
\subsection{A Suggestion:}\label{a-suggestion}}

Can we do something to make public domain texts more useful? Is there a
place for (some) scholars to take the lead here? Rather than paying
Kessinger to print out Google Books page-scans, could we not use the (in
this case, multiple sets of) page-scans available from a variety of
sources to put together a lightly marked up version of the text?
Couldn't we draw on existing bibliographies to make clear what the book
object represented by those scans actually is. And then, from our single
encoding, could we not export to multiple formats: PDF (by way of LaTeX,
for folks who want to print this thing out); HTML; and ePub (etc) for
eReaders?

Such an idea is not novel; it is merely an expression of the dream of a
markup language like TEI. Not so long ago, a proposal for a
\href{http://neosmart.net/blog/2012/the-case-for-a-git-powered-project-gutenberg/}{``A
Git Powered Project Gutenberg''} lead to a
\href{http://news.ycombinator.com/item?id=3638917}{discussion on Hacker
News} which in turn lead to a
\href{http://groups.google.com/group/prj-alexandria}{hastily arranged
group} (which just as quickly disarranged itself)---all focused around
the idea of making public domain texts better. There is interest in
improving the accessibility and usability of public domain texts and it
isn't confined to academic literature departments.

Scholars could play a key role here by helping to establish a good text
and providing annotations and glosses or other contextual material. In
my wilder moments I imagine scholars providing a base text which than
then becomes the staple, raw ingredient in a variety of remix editions,
produced for audiences varying from high school to the college
classroom, and beyond. These texts in turn could be cut and remixed to
produce a roll-your-own anthology.

\hypertarget{an-acknowledgment-and-a-goal}{%
\subsection{An Acknowledgment and a
Goal:}\label{an-acknowledgment-and-a-goal}}

There are some excellent reasons why \textbf{I} shouldn't be doing this.
First, in the specific case of \emph{Harlem Shadows}, I am not a
specialist in American, African American, or Caribbean literature in
general, nor in Claude McKay's work in particular. Nor am I an expert in
text markup. Nor am I sufficiently well versed in the dark
bibliographical arts to really be handling the complexities of putting
together a proper critical edition.

With those reservations stated, I'm trying to carve some time out to
work on this nonetheless. One's reach should exceed one's grasp, else
what's a public domain for? But boy would I love some help.

I've converted the plaintext, OCR'd version of \emph{Harlem Shadows}
available through archive.org to a lightly marked up TEI version of that
text. This markup itself is worthy of scrutiny; but I wanted to have
\emph{something} to start with on the way to producing a proofread,
bibliographically sound, TEI-version of the text; to that I'd like to
add annotations and textual notes, as well as supplementary
material---early reviews, maybe McKay's prose from this period, as
relevant. Think Norton Critical Edition (minus the criticism which is
likely too thorny a permissions matter; though I'd love to proved wrong
on this front).

To begin:

\begin{itemize}
\tightlist
\item
  here is a \href{http://github.com/c-forster/harlem-shadows}{github
  repository} with my initial stab at marking up the text.
\item
  here is a \href{http://harlemshadows.pbworks.com}{wiki} to organize
  future work. (Let me know if you want to be added to the wiki).
\end{itemize}

(A minor technical note: For a while I was imagining that it would be
possible to use stand-off markup to keep text and annotation completely
separate. This would be great for many reasons; in theory, one could
have different sets of notes for different audiences (the high school
versus the college class room; a reading versus a scholarly edition);
from the little reading I've done, that seems not easily feasible at the
moment. For software developers, however, the problem of how to combine
constantly evolving sets of dependent texts is simply a fact of life;
version control systems, like git, provide some help in managing this
problem.)

As a preliminary schedule: begin finalizing markup of the edition by the
end of the summer. Continue collecting and adding supplementary material
and annotations in the Fall. Then start working on processing the text
out to desired formats (the
\href{http://www.tei-c.org/Tools/Stylesheets/}{TEI Stylesheets} provide
a great place to start); so that this time next summer, an edition of
sorts (available in multiple formats) is done.

For now I'd be interested in other folks sharing their thoughts,
criticism, or enthusiasm. Or, better yet, take some of this material and
fix it or fork it.

\end{document}
